\documentclass[11pt]{article}

    \usepackage[breakable]{tcolorbox}
    \usepackage{parskip} % Stop auto-indenting (to mimic markdown behaviour)
    
    \usepackage{iftex}
    \usepackage[russian,english]{babel}
    \usepackage{fontspec}

    \setsansfont{PT Sans}
    \setmainfont{PT Serif}
    \setmonofont{PT Mono}

    % Basic figure setup, for now with no caption control since it's done
    % automatically by Pandoc (which extracts ![](path) syntax from Markdown).
    \usepackage{graphicx}
    % Maintain compatibility with old templates. Remove in nbconvert 6.0
    \let\Oldincludegraphics\includegraphics
    % Ensure that by default, figures have no caption (until we provide a
    % proper Figure object with a Caption API and a way to capture that
    % in the conversion process - todo).
    \usepackage{caption}
    \DeclareCaptionFormat{nocaption}{}
    \captionsetup{format=nocaption,aboveskip=0pt,belowskip=0pt}

    \usepackage{float}
    \floatplacement{figure}{H} % forces figures to be placed at the correct location
    \usepackage{xcolor} % Allow colors to be defined
    \usepackage{enumerate} % Needed for markdown enumerations to work
    \usepackage{geometry} % Used to adjust the document margins
    \usepackage{amsmath} % Equations
    \usepackage{amssymb} % Equations
    \usepackage{textcomp} % defines textquotesingle
    % Hack from http://tex.stackexchange.com/a/47451/13684:
    \AtBeginDocument{%
        \def\PYZsq{\textquotesingle}% Upright quotes in Pygmentized code
    }
    \usepackage{upquote} % Upright quotes for verbatim code
    \usepackage{eurosym} % defines \euro
    \usepackage[mathletters]{ucs} % Extended unicode (utf-8) support
    \usepackage{fancyvrb} % verbatim replacement that allows latex
    \usepackage{grffile} % extends the file name processing of package graphics 
                         % to support a larger range
    \makeatletter % fix for old versions of grffile with XeLaTeX
    \@ifpackagelater{grffile}{2019/11/01}
    {
      % Do nothing on new versions
    }
    {
      \def\Gread@@xetex#1{%
        \IfFileExists{"\Gin@base".bb}%
        {\Gread@eps{\Gin@base.bb}}%
        {\Gread@@xetex@aux#1}%
      }
    }
    \makeatother
    \usepackage[Export]{adjustbox} % Used to constrain images to a maximum size
    \adjustboxset{max size={0.9\linewidth}{0.9\paperheight}}

    % The hyperref package gives us a pdf with properly built
    % internal navigation ('pdf bookmarks' for the table of contents,
    % internal cross-reference links, web links for URLs, etc.)
    \usepackage{hyperref}
    % The default LaTeX title has an obnoxious amount of whitespace. By default,
    % titling removes some of it. It also provides customization options.
    \usepackage{titling}
    \usepackage{longtable} % longtable support required by pandoc >1.10
    \usepackage{booktabs}  % table support for pandoc > 1.12.2
    \usepackage[inline]{enumitem} % IRkernel/repr support (it uses the enumerate* environment)
    \usepackage[normalem]{ulem} % ulem is needed to support strikethroughs (\sout)
                                % normalem makes italics be italics, not underlines
    \usepackage{mathrsfs}
    

    
    % Colors for the hyperref package
    \definecolor{urlcolor}{rgb}{0,.145,.698}
    \definecolor{linkcolor}{rgb}{.71,0.21,0.01}
    \definecolor{citecolor}{rgb}{.12,.54,.11}

    % ANSI colors
    \definecolor{ansi-black}{HTML}{3E424D}
    \definecolor{ansi-black-intense}{HTML}{282C36}
    \definecolor{ansi-red}{HTML}{E75C58}
    \definecolor{ansi-red-intense}{HTML}{B22B31}
    \definecolor{ansi-green}{HTML}{00A250}
    \definecolor{ansi-green-intense}{HTML}{007427}
    \definecolor{ansi-yellow}{HTML}{DDB62B}
    \definecolor{ansi-yellow-intense}{HTML}{B27D12}
    \definecolor{ansi-blue}{HTML}{208FFB}
    \definecolor{ansi-blue-intense}{HTML}{0065CA}
    \definecolor{ansi-magenta}{HTML}{D160C4}
    \definecolor{ansi-magenta-intense}{HTML}{A03196}
    \definecolor{ansi-cyan}{HTML}{60C6C8}
    \definecolor{ansi-cyan-intense}{HTML}{258F8F}
    \definecolor{ansi-white}{HTML}{C5C1B4}
    \definecolor{ansi-white-intense}{HTML}{A1A6B2}
    \definecolor{ansi-default-inverse-fg}{HTML}{FFFFFF}
    \definecolor{ansi-default-inverse-bg}{HTML}{000000}

    % common color for the border for error outputs.
    \definecolor{outerrorbackground}{HTML}{FFDFDF}

    % commands and environments needed by pandoc snippets
    % extracted from the output of `pandoc -s`
    \providecommand{\tightlist}{%
      \setlength{\itemsep}{0pt}\setlength{\parskip}{0pt}}
    \DefineVerbatimEnvironment{Highlighting}{Verbatim}{commandchars=\\\{\}}
    % Add ',fontsize=\small' for more characters per line
    \newenvironment{Shaded}{}{}
    \newcommand{\KeywordTok}[1]{\textcolor[rgb]{0.00,0.44,0.13}{\textbf{{#1}}}}
    \newcommand{\DataTypeTok}[1]{\textcolor[rgb]{0.56,0.13,0.00}{{#1}}}
    \newcommand{\DecValTok}[1]{\textcolor[rgb]{0.25,0.63,0.44}{{#1}}}
    \newcommand{\BaseNTok}[1]{\textcolor[rgb]{0.25,0.63,0.44}{{#1}}}
    \newcommand{\FloatTok}[1]{\textcolor[rgb]{0.25,0.63,0.44}{{#1}}}
    \newcommand{\CharTok}[1]{\textcolor[rgb]{0.25,0.44,0.63}{{#1}}}
    \newcommand{\StringTok}[1]{\textcolor[rgb]{0.25,0.44,0.63}{{#1}}}
    \newcommand{\CommentTok}[1]{\textcolor[rgb]{0.38,0.63,0.69}{\textit{{#1}}}}
    \newcommand{\OtherTok}[1]{\textcolor[rgb]{0.00,0.44,0.13}{{#1}}}
    \newcommand{\AlertTok}[1]{\textcolor[rgb]{1.00,0.00,0.00}{\textbf{{#1}}}}
    \newcommand{\FunctionTok}[1]{\textcolor[rgb]{0.02,0.16,0.49}{{#1}}}
    \newcommand{\RegionMarkerTok}[1]{{#1}}
    \newcommand{\ErrorTok}[1]{\textcolor[rgb]{1.00,0.00,0.00}{\textbf{{#1}}}}
    \newcommand{\NormalTok}[1]{{#1}}
    
    % Additional commands for more recent versions of Pandoc
    \newcommand{\ConstantTok}[1]{\textcolor[rgb]{0.53,0.00,0.00}{{#1}}}
    \newcommand{\SpecialCharTok}[1]{\textcolor[rgb]{0.25,0.44,0.63}{{#1}}}
    \newcommand{\VerbatimStringTok}[1]{\textcolor[rgb]{0.25,0.44,0.63}{{#1}}}
    \newcommand{\SpecialStringTok}[1]{\textcolor[rgb]{0.73,0.40,0.53}{{#1}}}
    \newcommand{\ImportTok}[1]{{#1}}
    \newcommand{\DocumentationTok}[1]{\textcolor[rgb]{0.73,0.13,0.13}{\textit{{#1}}}}
    \newcommand{\AnnotationTok}[1]{\textcolor[rgb]{0.38,0.63,0.69}{\textbf{\textit{{#1}}}}}
    \newcommand{\CommentVarTok}[1]{\textcolor[rgb]{0.38,0.63,0.69}{\textbf{\textit{{#1}}}}}
    \newcommand{\VariableTok}[1]{\textcolor[rgb]{0.10,0.09,0.49}{{#1}}}
    \newcommand{\ControlFlowTok}[1]{\textcolor[rgb]{0.00,0.44,0.13}{\textbf{{#1}}}}
    \newcommand{\OperatorTok}[1]{\textcolor[rgb]{0.40,0.40,0.40}{{#1}}}
    \newcommand{\BuiltInTok}[1]{{#1}}
    \newcommand{\ExtensionTok}[1]{{#1}}
    \newcommand{\PreprocessorTok}[1]{\textcolor[rgb]{0.74,0.48,0.00}{{#1}}}
    \newcommand{\AttributeTok}[1]{\textcolor[rgb]{0.49,0.56,0.16}{{#1}}}
    \newcommand{\InformationTok}[1]{\textcolor[rgb]{0.38,0.63,0.69}{\textbf{\textit{{#1}}}}}
    \newcommand{\WarningTok}[1]{\textcolor[rgb]{0.38,0.63,0.69}{\textbf{\textit{{#1}}}}}
    
    
    % Define a nice break command that doesn't care if a line doesn't already
    % exist.
    \def\br{\hspace*{\fill} \\* }
    % Math Jax compatibility definitions
    \def\gt{>}
    \def\lt{<}
    \let\Oldtex\TeX
    \let\Oldlatex\LaTeX
    \renewcommand{\TeX}{\textrm{\Oldtex}}
    \renewcommand{\LaTeX}{\textrm{\Oldlatex}}
    % Document parameters
    % Document title
    \title{Разведочный анализ данных. Исследование и визуализация данных.}
    
    
    
    \author{Алпеев Владислав Сергеевич}
    
    
    
% Pygments definitions
\makeatletter
\def\PY@reset{\let\PY@it=\relax \let\PY@bf=\relax%
    \let\PY@ul=\relax \let\PY@tc=\relax%
    \let\PY@bc=\relax \let\PY@ff=\relax}
\def\PY@tok#1{\csname PY@tok@#1\endcsname}
\def\PY@toks#1+{\ifx\relax#1\empty\else%
    \PY@tok{#1}\expandafter\PY@toks\fi}
\def\PY@do#1{\PY@bc{\PY@tc{\PY@ul{%
    \PY@it{\PY@bf{\PY@ff{#1}}}}}}}
\def\PY#1#2{\PY@reset\PY@toks#1+\relax+\PY@do{#2}}

\expandafter\def\csname PY@tok@w\endcsname{\def\PY@tc##1{\textcolor[rgb]{0.73,0.73,0.73}{##1}}}
\expandafter\def\csname PY@tok@c\endcsname{\let\PY@it=\textit\def\PY@tc##1{\textcolor[rgb]{0.25,0.50,0.50}{##1}}}
\expandafter\def\csname PY@tok@cp\endcsname{\def\PY@tc##1{\textcolor[rgb]{0.74,0.48,0.00}{##1}}}
\expandafter\def\csname PY@tok@k\endcsname{\let\PY@bf=\textbf\def\PY@tc##1{\textcolor[rgb]{0.00,0.50,0.00}{##1}}}
\expandafter\def\csname PY@tok@kp\endcsname{\def\PY@tc##1{\textcolor[rgb]{0.00,0.50,0.00}{##1}}}
\expandafter\def\csname PY@tok@kt\endcsname{\def\PY@tc##1{\textcolor[rgb]{0.69,0.00,0.25}{##1}}}
\expandafter\def\csname PY@tok@o\endcsname{\def\PY@tc##1{\textcolor[rgb]{0.40,0.40,0.40}{##1}}}
\expandafter\def\csname PY@tok@ow\endcsname{\let\PY@bf=\textbf\def\PY@tc##1{\textcolor[rgb]{0.67,0.13,1.00}{##1}}}
\expandafter\def\csname PY@tok@nb\endcsname{\def\PY@tc##1{\textcolor[rgb]{0.00,0.50,0.00}{##1}}}
\expandafter\def\csname PY@tok@nf\endcsname{\def\PY@tc##1{\textcolor[rgb]{0.00,0.00,1.00}{##1}}}
\expandafter\def\csname PY@tok@nc\endcsname{\let\PY@bf=\textbf\def\PY@tc##1{\textcolor[rgb]{0.00,0.00,1.00}{##1}}}
\expandafter\def\csname PY@tok@nn\endcsname{\let\PY@bf=\textbf\def\PY@tc##1{\textcolor[rgb]{0.00,0.00,1.00}{##1}}}
\expandafter\def\csname PY@tok@ne\endcsname{\let\PY@bf=\textbf\def\PY@tc##1{\textcolor[rgb]{0.82,0.25,0.23}{##1}}}
\expandafter\def\csname PY@tok@nv\endcsname{\def\PY@tc##1{\textcolor[rgb]{0.10,0.09,0.49}{##1}}}
\expandafter\def\csname PY@tok@no\endcsname{\def\PY@tc##1{\textcolor[rgb]{0.53,0.00,0.00}{##1}}}
\expandafter\def\csname PY@tok@nl\endcsname{\def\PY@tc##1{\textcolor[rgb]{0.63,0.63,0.00}{##1}}}
\expandafter\def\csname PY@tok@ni\endcsname{\let\PY@bf=\textbf\def\PY@tc##1{\textcolor[rgb]{0.60,0.60,0.60}{##1}}}
\expandafter\def\csname PY@tok@na\endcsname{\def\PY@tc##1{\textcolor[rgb]{0.49,0.56,0.16}{##1}}}
\expandafter\def\csname PY@tok@nt\endcsname{\let\PY@bf=\textbf\def\PY@tc##1{\textcolor[rgb]{0.00,0.50,0.00}{##1}}}
\expandafter\def\csname PY@tok@nd\endcsname{\def\PY@tc##1{\textcolor[rgb]{0.67,0.13,1.00}{##1}}}
\expandafter\def\csname PY@tok@s\endcsname{\def\PY@tc##1{\textcolor[rgb]{0.73,0.13,0.13}{##1}}}
\expandafter\def\csname PY@tok@sd\endcsname{\let\PY@it=\textit\def\PY@tc##1{\textcolor[rgb]{0.73,0.13,0.13}{##1}}}
\expandafter\def\csname PY@tok@si\endcsname{\let\PY@bf=\textbf\def\PY@tc##1{\textcolor[rgb]{0.73,0.40,0.53}{##1}}}
\expandafter\def\csname PY@tok@se\endcsname{\let\PY@bf=\textbf\def\PY@tc##1{\textcolor[rgb]{0.73,0.40,0.13}{##1}}}
\expandafter\def\csname PY@tok@sr\endcsname{\def\PY@tc##1{\textcolor[rgb]{0.73,0.40,0.53}{##1}}}
\expandafter\def\csname PY@tok@ss\endcsname{\def\PY@tc##1{\textcolor[rgb]{0.10,0.09,0.49}{##1}}}
\expandafter\def\csname PY@tok@sx\endcsname{\def\PY@tc##1{\textcolor[rgb]{0.00,0.50,0.00}{##1}}}
\expandafter\def\csname PY@tok@m\endcsname{\def\PY@tc##1{\textcolor[rgb]{0.40,0.40,0.40}{##1}}}
\expandafter\def\csname PY@tok@gh\endcsname{\let\PY@bf=\textbf\def\PY@tc##1{\textcolor[rgb]{0.00,0.00,0.50}{##1}}}
\expandafter\def\csname PY@tok@gu\endcsname{\let\PY@bf=\textbf\def\PY@tc##1{\textcolor[rgb]{0.50,0.00,0.50}{##1}}}
\expandafter\def\csname PY@tok@gd\endcsname{\def\PY@tc##1{\textcolor[rgb]{0.63,0.00,0.00}{##1}}}
\expandafter\def\csname PY@tok@gi\endcsname{\def\PY@tc##1{\textcolor[rgb]{0.00,0.63,0.00}{##1}}}
\expandafter\def\csname PY@tok@gr\endcsname{\def\PY@tc##1{\textcolor[rgb]{1.00,0.00,0.00}{##1}}}
\expandafter\def\csname PY@tok@ge\endcsname{\let\PY@it=\textit}
\expandafter\def\csname PY@tok@gs\endcsname{\let\PY@bf=\textbf}
\expandafter\def\csname PY@tok@gp\endcsname{\let\PY@bf=\textbf\def\PY@tc##1{\textcolor[rgb]{0.00,0.00,0.50}{##1}}}
\expandafter\def\csname PY@tok@go\endcsname{\def\PY@tc##1{\textcolor[rgb]{0.53,0.53,0.53}{##1}}}
\expandafter\def\csname PY@tok@gt\endcsname{\def\PY@tc##1{\textcolor[rgb]{0.00,0.27,0.87}{##1}}}
\expandafter\def\csname PY@tok@err\endcsname{\def\PY@bc##1{\setlength{\fboxsep}{0pt}\fcolorbox[rgb]{1.00,0.00,0.00}{1,1,1}{\strut ##1}}}
\expandafter\def\csname PY@tok@kc\endcsname{\let\PY@bf=\textbf\def\PY@tc##1{\textcolor[rgb]{0.00,0.50,0.00}{##1}}}
\expandafter\def\csname PY@tok@kd\endcsname{\let\PY@bf=\textbf\def\PY@tc##1{\textcolor[rgb]{0.00,0.50,0.00}{##1}}}
\expandafter\def\csname PY@tok@kn\endcsname{\let\PY@bf=\textbf\def\PY@tc##1{\textcolor[rgb]{0.00,0.50,0.00}{##1}}}
\expandafter\def\csname PY@tok@kr\endcsname{\let\PY@bf=\textbf\def\PY@tc##1{\textcolor[rgb]{0.00,0.50,0.00}{##1}}}
\expandafter\def\csname PY@tok@bp\endcsname{\def\PY@tc##1{\textcolor[rgb]{0.00,0.50,0.00}{##1}}}
\expandafter\def\csname PY@tok@fm\endcsname{\def\PY@tc##1{\textcolor[rgb]{0.00,0.00,1.00}{##1}}}
\expandafter\def\csname PY@tok@vc\endcsname{\def\PY@tc##1{\textcolor[rgb]{0.10,0.09,0.49}{##1}}}
\expandafter\def\csname PY@tok@vg\endcsname{\def\PY@tc##1{\textcolor[rgb]{0.10,0.09,0.49}{##1}}}
\expandafter\def\csname PY@tok@vi\endcsname{\def\PY@tc##1{\textcolor[rgb]{0.10,0.09,0.49}{##1}}}
\expandafter\def\csname PY@tok@vm\endcsname{\def\PY@tc##1{\textcolor[rgb]{0.10,0.09,0.49}{##1}}}
\expandafter\def\csname PY@tok@sa\endcsname{\def\PY@tc##1{\textcolor[rgb]{0.73,0.13,0.13}{##1}}}
\expandafter\def\csname PY@tok@sb\endcsname{\def\PY@tc##1{\textcolor[rgb]{0.73,0.13,0.13}{##1}}}
\expandafter\def\csname PY@tok@sc\endcsname{\def\PY@tc##1{\textcolor[rgb]{0.73,0.13,0.13}{##1}}}
\expandafter\def\csname PY@tok@dl\endcsname{\def\PY@tc##1{\textcolor[rgb]{0.73,0.13,0.13}{##1}}}
\expandafter\def\csname PY@tok@s2\endcsname{\def\PY@tc##1{\textcolor[rgb]{0.73,0.13,0.13}{##1}}}
\expandafter\def\csname PY@tok@sh\endcsname{\def\PY@tc##1{\textcolor[rgb]{0.73,0.13,0.13}{##1}}}
\expandafter\def\csname PY@tok@s1\endcsname{\def\PY@tc##1{\textcolor[rgb]{0.73,0.13,0.13}{##1}}}
\expandafter\def\csname PY@tok@mb\endcsname{\def\PY@tc##1{\textcolor[rgb]{0.40,0.40,0.40}{##1}}}
\expandafter\def\csname PY@tok@mf\endcsname{\def\PY@tc##1{\textcolor[rgb]{0.40,0.40,0.40}{##1}}}
\expandafter\def\csname PY@tok@mh\endcsname{\def\PY@tc##1{\textcolor[rgb]{0.40,0.40,0.40}{##1}}}
\expandafter\def\csname PY@tok@mi\endcsname{\def\PY@tc##1{\textcolor[rgb]{0.40,0.40,0.40}{##1}}}
\expandafter\def\csname PY@tok@il\endcsname{\def\PY@tc##1{\textcolor[rgb]{0.40,0.40,0.40}{##1}}}
\expandafter\def\csname PY@tok@mo\endcsname{\def\PY@tc##1{\textcolor[rgb]{0.40,0.40,0.40}{##1}}}
\expandafter\def\csname PY@tok@ch\endcsname{\let\PY@it=\textit\def\PY@tc##1{\textcolor[rgb]{0.25,0.50,0.50}{##1}}}
\expandafter\def\csname PY@tok@cm\endcsname{\let\PY@it=\textit\def\PY@tc##1{\textcolor[rgb]{0.25,0.50,0.50}{##1}}}
\expandafter\def\csname PY@tok@cpf\endcsname{\let\PY@it=\textit\def\PY@tc##1{\textcolor[rgb]{0.25,0.50,0.50}{##1}}}
\expandafter\def\csname PY@tok@c1\endcsname{\let\PY@it=\textit\def\PY@tc##1{\textcolor[rgb]{0.25,0.50,0.50}{##1}}}
\expandafter\def\csname PY@tok@cs\endcsname{\let\PY@it=\textit\def\PY@tc##1{\textcolor[rgb]{0.25,0.50,0.50}{##1}}}

\def\PYZbs{\char`\\}
\def\PYZus{\char`\_}
\def\PYZob{\char`\{}
\def\PYZcb{\char`\}}
\def\PYZca{\char`\^}
\def\PYZam{\char`\&}
\def\PYZlt{\char`\<}
\def\PYZgt{\char`\>}
\def\PYZsh{\char`\#}
\def\PYZpc{\char`\%}
\def\PYZdl{\char`\$}
\def\PYZhy{\char`\-}
\def\PYZsq{\char`\'}
\def\PYZdq{\char`\"}
\def\PYZti{\char`\~}
% for compatibility with earlier versions
\def\PYZat{@}
\def\PYZlb{[}
\def\PYZrb{]}
\makeatother


    % For linebreaks inside Verbatim environment from package fancyvrb. 
    \makeatletter
        \newbox\Wrappedcontinuationbox 
        \newbox\Wrappedvisiblespacebox 
        \newcommand*\Wrappedvisiblespace {\textcolor{red}{\textvisiblespace}} 
        \newcommand*\Wrappedcontinuationsymbol {\textcolor{red}{\llap{\tiny$\m@th\hookrightarrow$}}} 
        \newcommand*\Wrappedcontinuationindent {3ex } 
        \newcommand*\Wrappedafterbreak {\kern\Wrappedcontinuationindent\copy\Wrappedcontinuationbox} 
        % Take advantage of the already applied Pygments mark-up to insert 
        % potential linebreaks for TeX processing. 
        %        {, <, #, %, $, ' and ": go to next line. 
        %        _, }, ^, &, >, - and ~: stay at end of broken line. 
        % Use of \textquotesingle for straight quote. 
        \newcommand*\Wrappedbreaksatspecials {% 
            \def\PYGZus{\discretionary{\char`\_}{\Wrappedafterbreak}{\char`\_}}% 
            \def\PYGZob{\discretionary{}{\Wrappedafterbreak\char`\{}{\char`\{}}% 
            \def\PYGZcb{\discretionary{\char`\}}{\Wrappedafterbreak}{\char`\}}}% 
            \def\PYGZca{\discretionary{\char`\^}{\Wrappedafterbreak}{\char`\^}}% 
            \def\PYGZam{\discretionary{\char`\&}{\Wrappedafterbreak}{\char`\&}}% 
            \def\PYGZlt{\discretionary{}{\Wrappedafterbreak\char`\<}{\char`\<}}% 
            \def\PYGZgt{\discretionary{\char`\>}{\Wrappedafterbreak}{\char`\>}}% 
            \def\PYGZsh{\discretionary{}{\Wrappedafterbreak\char`\#}{\char`\#}}% 
            \def\PYGZpc{\discretionary{}{\Wrappedafterbreak\char`\%}{\char`\%}}% 
            \def\PYGZdl{\discretionary{}{\Wrappedafterbreak\char`\$}{\char`\$}}% 
            \def\PYGZhy{\discretionary{\char`\-}{\Wrappedafterbreak}{\char`\-}}% 
            \def\PYGZsq{\discretionary{}{\Wrappedafterbreak\textquotesingle}{\textquotesingle}}% 
            \def\PYGZdq{\discretionary{}{\Wrappedafterbreak\char`\"}{\char`\"}}% 
            \def\PYGZti{\discretionary{\char`\~}{\Wrappedafterbreak}{\char`\~}}% 
        } 
        % Some characters . , ; ? ! / are not pygmentized. 
        % This macro makes them "active" and they will insert potential linebreaks 
        \newcommand*\Wrappedbreaksatpunct {% 
            \lccode`\~`\.\lowercase{\def~}{\discretionary{\hbox{\char`\.}}{\Wrappedafterbreak}{\hbox{\char`\.}}}% 
            \lccode`\~`\,\lowercase{\def~}{\discretionary{\hbox{\char`\,}}{\Wrappedafterbreak}{\hbox{\char`\,}}}% 
            \lccode`\~`\;\lowercase{\def~}{\discretionary{\hbox{\char`\;}}{\Wrappedafterbreak}{\hbox{\char`\;}}}% 
            \lccode`\~`\:\lowercase{\def~}{\discretionary{\hbox{\char`\:}}{\Wrappedafterbreak}{\hbox{\char`\:}}}% 
            \lccode`\~`\?\lowercase{\def~}{\discretionary{\hbox{\char`\?}}{\Wrappedafterbreak}{\hbox{\char`\?}}}% 
            \lccode`\~`\!\lowercase{\def~}{\discretionary{\hbox{\char`\!}}{\Wrappedafterbreak}{\hbox{\char`\!}}}% 
            \lccode`\~`\/\lowercase{\def~}{\discretionary{\hbox{\char`\/}}{\Wrappedafterbreak}{\hbox{\char`\/}}}% 
            \catcode`\.\active
            \catcode`\,\active 
            \catcode`\;\active
            \catcode`\:\active
            \catcode`\?\active
            \catcode`\!\active
            \catcode`\/\active 
            \lccode`\~`\~ 	
        }
    \makeatother

    \let\OriginalVerbatim=\Verbatim
    \makeatletter
    \renewcommand{\Verbatim}[1][1]{%
        %\parskip\z@skip
        \sbox\Wrappedcontinuationbox {\Wrappedcontinuationsymbol}%
        \sbox\Wrappedvisiblespacebox {\FV@SetupFont\Wrappedvisiblespace}%
        \def\FancyVerbFormatLine ##1{\hsize\linewidth
            \vtop{\raggedright\hyphenpenalty\z@\exhyphenpenalty\z@
                \doublehyphendemerits\z@\finalhyphendemerits\z@
                \strut ##1\strut}%
        }%
        % If the linebreak is at a space, the latter will be displayed as visible
        % space at end of first line, and a continuation symbol starts next line.
        % Stretch/shrink are however usually zero for typewriter font.
        \def\FV@Space {%
            \nobreak\hskip\z@ plus\fontdimen3\font minus\fontdimen4\font
            \discretionary{\copy\Wrappedvisiblespacebox}{\Wrappedafterbreak}
            {\kern\fontdimen2\font}%
        }%
        
        % Allow breaks at special characters using \PYG... macros.
        \Wrappedbreaksatspecials
        % Breaks at punctuation characters . , ; ? ! and / need catcode=\active 	
        \OriginalVerbatim[#1,codes*=\Wrappedbreaksatpunct]%
    }
    \makeatother

    % Exact colors from NB
    \definecolor{incolor}{HTML}{303F9F}
    \definecolor{outcolor}{HTML}{D84315}
    \definecolor{cellborder}{HTML}{CFCFCF}
    \definecolor{cellbackground}{HTML}{F7F7F7}
    
    % prompt
    \makeatletter
    \newcommand{\boxspacing}{\kern\kvtcb@left@rule\kern\kvtcb@boxsep}
    \makeatother
    \newcommand{\prompt}[4]{
        {\ttfamily\llap{{\color{#2}[#3]:\hspace{3pt}#4}}\vspace{-\baselineskip}}
    }
    

    
    % Prevent overflowing lines due to hard-to-break entities
    \sloppy 
    % Setup hyperref package
    \hypersetup{
      breaklinks=true,  % so long urls are correctly broken across lines
      colorlinks=true,
      urlcolor=urlcolor,
      linkcolor=linkcolor,
      citecolor=citecolor,
      }
    % Slightly bigger margins than the latex defaults
    
    \geometry{verbose,tmargin=1in,bmargin=1in,lmargin=1in,rmargin=1in}
    
    

\begin{document}
    
    \maketitle
    
    

    
    Разведочный анализ данных. Исследование и визуализация данных.

\begin{enumerate}
\def\labelenumi{\arabic{enumi}.}
\tightlist
\item
  Цель лабораторной работы

  Изучение различных методов визуализация данных.

  \begin{enumerate}
  \def\labelenumii{\arabic{enumii}.}
  \setcounter{enumii}{1}
  \tightlist
  \item
    Краткое описание.

    Построение основных графиков, входящих в этап разведочного анализа
    данных.

    \begin{enumerate}
    \def\labelenumiii{\arabic{enumiii}.}
    \setcounter{enumiii}{2}
    \tightlist
    \item
      Задание.

      Выбрать набор данных (датасет).

      Создать ноутбук, который содержит следующие разделы:

      Текстовое описание выбранного Вами набора данных.

      Основные характеристики датасета.

      Визуальное исследование датасета.

      Информация о корреляции признаков

      Сформировать отчет и разместить его в своем репозитории на github.

      \begin{enumerate}
      \def\labelenumiv{\arabic{enumiv}.}
      \setcounter{enumiv}{3}
      \tightlist
      \item
        Выполение работы.

        Текстовое описание датасета.
      \end{enumerate}
    \end{enumerate}
  \end{enumerate}
\end{enumerate}

    В качестве набора данных мы будем использовать набор данных о
характеристиках вина -
https://scikit-learn.org/stable/modules/generated/sklearn.datasets.load\_wine.html\#sklearn.datasets.load\_wine

Представлоенная база данных содержит информацию важную для
сравнительного анализа содержания вина, выращенного на разных
винодельнях. Благодаря обработке этих данных можно сделать выводы о том,
к какому результату приводит использование той или иной технологии
производства.

Датасет состоит из одного файла wine.data.

Файл содержит следующие колонки, обозначающие элементы, содержащиеся в
вине:

\begin{itemize}
\tightlist
\item
  Alcohol - содержание алкоголя в \%.
\item
  Malic acid - яблочная кислота.
\item
  Ash - зола в мг/л.
\item
  Alcalinity of ash - щелочность золы.
\item
  Magnesium - содержание магния в mg.
\item
  Total phenols - количество фенолов.
\item
  Flavanoids - количество флаваноидов.
\item
  Nonflavanoid phenols - количество нефлаваноидных фенолов
\item
  Proanthocyanins - проантоцианы.
\item
  Color intensity - интенсивность цвета.
\item
  Hue - оттенок.
\item
  OD280/OD315 of diluted wines - содержание этих элементов в
  разбавленном вине.
\item
  Proline - пролин.
\end{itemize}

Импортируем необходимые библиотеки и загрузим необходимую базу даннных
из sklearn при помощи соответствующей функции.

    \begin{tcolorbox}[breakable, size=fbox, boxrule=1pt, pad at break*=1mm,colback=cellbackground, colframe=cellborder]
\prompt{In}{incolor}{88}{\boxspacing}
\begin{Verbatim}[commandchars=\\\{\}]
\PY{k+kn}{import} \PY{n+nn}{numpy} \PY{k}{as} \PY{n+nn}{np}
\PY{k+kn}{import} \PY{n+nn}{pandas} \PY{k}{as} \PY{n+nn}{pd} \PY{c+c1}{\PYZsh{}оформление таблицы}
\PY{k+kn}{from} \PY{n+nn}{sklearn}\PY{n+nn}{.}\PY{n+nn}{datasets} \PY{k+kn}{import} \PY{o}{*} \PY{c+c1}{\PYZsh{}получение датасета}
\PY{k+kn}{import} \PY{n+nn}{matplotlib}\PY{n+nn}{.}\PY{n+nn}{pyplot} \PY{k}{as} \PY{n+nn}{plt}  \PY{c+c1}{\PYZsh{}создание графиков}
\PY{k+kn}{import} \PY{n+nn}{seaborn} \PY{k}{as} \PY{n+nn}{sns} \PY{c+c1}{\PYZsh{}визуализация данных}
\PY{n}{sns}\PY{o}{.}\PY{n}{set}\PY{p}{(}\PY{n}{style}\PY{o}{=}\PY{l+s+s2}{\PYZdq{}}\PY{l+s+s2}{ticks}\PY{l+s+s2}{\PYZdq{}}\PY{p}{)}

\PY{n}{wine} \PY{o}{=} \PY{n}{load\PYZus{}wine}\PY{p}{(}\PY{p}{)}
\end{Verbatim}
\end{tcolorbox}

    Основные характеристики датасета.

    Здесь представлены 13 различных измерений полученных для различных
состоявляющих, найденных в вине.

    \begin{tcolorbox}[breakable, size=fbox, boxrule=1pt, pad at break*=1mm,colback=cellbackground, colframe=cellborder]
\prompt{In}{incolor}{57}{\boxspacing}
\begin{Verbatim}[commandchars=\\\{\}]
\PY{n}{wine}\PY{p}{[}\PY{l+s+s1}{\PYZsq{}}\PY{l+s+s1}{feature\PYZus{}names}\PY{l+s+s1}{\PYZsq{}}\PY{p}{]}
\end{Verbatim}
\end{tcolorbox}

            \begin{tcolorbox}[breakable, size=fbox, boxrule=.5pt, pad at break*=1mm, opacityfill=0]
\prompt{Out}{outcolor}{57}{\boxspacing}
\begin{Verbatim}[commandchars=\\\{\}]
['alcohol',
 'malic\_acid',
 'ash',
 'alcalinity\_of\_ash',
 'magnesium',
 'total\_phenols',
 'flavanoids',
 'nonflavanoid\_phenols',
 'proanthocyanins',
 'color\_intensity',
 'hue',
 'od280/od315\_of\_diluted\_wines',
 'proline']
\end{Verbatim}
\end{tcolorbox}
        
    \begin{tcolorbox}[breakable, size=fbox, boxrule=1pt, pad at break*=1mm,colback=cellbackground, colframe=cellborder]
\prompt{In}{incolor}{59}{\boxspacing}
\begin{Verbatim}[commandchars=\\\{\}]
\PY{n}{wine}\PY{p}{[}\PY{l+s+s1}{\PYZsq{}}\PY{l+s+s1}{target\PYZus{}names}\PY{l+s+s1}{\PYZsq{}}\PY{p}{]}
\end{Verbatim}
\end{tcolorbox}

            \begin{tcolorbox}[breakable, size=fbox, boxrule=.5pt, pad at break*=1mm, opacityfill=0]
\prompt{Out}{outcolor}{59}{\boxspacing}
\begin{Verbatim}[commandchars=\\\{\}]
array(['class\_0', 'class\_1', 'class\_2'], dtype='<U7')
\end{Verbatim}
\end{tcolorbox}
        
    Здесь под классами подразумеваются три различных культиватора.

    Рассмотрим ряд первых записей базы данных:

    \begin{tcolorbox}[breakable, size=fbox, boxrule=1pt, pad at break*=1mm,colback=cellbackground, colframe=cellborder]
\prompt{In}{incolor}{33}{\boxspacing}
\begin{Verbatim}[commandchars=\\\{\}]
\PY{n}{data1} \PY{o}{=} \PY{n}{pd}\PY{o}{.}\PY{n}{DataFrame}\PY{p}{(}\PY{n}{data}\PY{o}{=} \PY{n}{np}\PY{o}{.}\PY{n}{c\PYZus{}}\PY{p}{[}\PY{n}{wine}\PY{p}{[}\PY{l+s+s1}{\PYZsq{}}\PY{l+s+s1}{data}\PY{l+s+s1}{\PYZsq{}}\PY{p}{]}\PY{p}{,} \PY{n}{wine}\PY{p}{[}\PY{l+s+s1}{\PYZsq{}}\PY{l+s+s1}{target}\PY{l+s+s1}{\PYZsq{}}\PY{p}{]}\PY{p}{]}\PY{p}{,}
                     \PY{n}{columns}\PY{o}{=} \PY{n}{wine}\PY{p}{[}\PY{l+s+s1}{\PYZsq{}}\PY{l+s+s1}{feature\PYZus{}names}\PY{l+s+s1}{\PYZsq{}}\PY{p}{]} \PY{o}{+} \PY{p}{[}\PY{l+s+s1}{\PYZsq{}}\PY{l+s+s1}{target}\PY{l+s+s1}{\PYZsq{}}\PY{p}{]}\PY{p}{)}
\PY{n}{data1}\PY{o}{.}\PY{n}{head}\PY{p}{(}\PY{p}{)}
\end{Verbatim}
\end{tcolorbox}

            \begin{tcolorbox}[breakable, size=fbox, boxrule=.5pt, pad at break*=1mm, opacityfill=0]
\prompt{Out}{outcolor}{33}{\boxspacing}
\begin{Verbatim}[commandchars=\\\{\}]
   alcohol  malic\_acid   ash  alcalinity\_of\_ash  magnesium  total\_phenols  \textbackslash{}
0    14.23        1.71  2.43               15.6      127.0           2.80
1    13.20        1.78  2.14               11.2      100.0           2.65
2    13.16        2.36  2.67               18.6      101.0           2.80
3    14.37        1.95  2.50               16.8      113.0           3.85
4    13.24        2.59  2.87               21.0      118.0           2.80

   flavanoids  nonflavanoid\_phenols  proanthocyanins  color\_intensity   hue  \textbackslash{}
0        3.06                  0.28             2.29             5.64  1.04
1        2.76                  0.26             1.28             4.38  1.05
2        3.24                  0.30             2.81             5.68  1.03
3        3.49                  0.24             2.18             7.80  0.86
4        2.69                  0.39             1.82             4.32  1.04

   od280/od315\_of\_diluted\_wines  proline  target
0                          3.92   1065.0     0.0
1                          3.40   1050.0     0.0
2                          3.17   1185.0     0.0
3                          3.45   1480.0     0.0
4                          2.93    735.0     0.0
\end{Verbatim}
\end{tcolorbox}
        
    Размер датасета - 178 строк и 13 столбцов.

    \begin{tcolorbox}[breakable, size=fbox, boxrule=1pt, pad at break*=1mm,colback=cellbackground, colframe=cellborder]
\prompt{In}{incolor}{34}{\boxspacing}
\begin{Verbatim}[commandchars=\\\{\}]
\PY{n}{wine}\PY{p}{[}\PY{l+s+s1}{\PYZsq{}}\PY{l+s+s1}{data}\PY{l+s+s1}{\PYZsq{}}\PY{p}{]}\PY{o}{.}\PY{n}{shape}
\end{Verbatim}
\end{tcolorbox}

            \begin{tcolorbox}[breakable, size=fbox, boxrule=.5pt, pad at break*=1mm, opacityfill=0]
\prompt{Out}{outcolor}{34}{\boxspacing}
\begin{Verbatim}[commandchars=\\\{\}]
(178, 13)
\end{Verbatim}
\end{tcolorbox}
        
    \begin{tcolorbox}[breakable, size=fbox, boxrule=1pt, pad at break*=1mm,colback=cellbackground, colframe=cellborder]
\prompt{In}{incolor}{35}{\boxspacing}
\begin{Verbatim}[commandchars=\\\{\}]
\PY{n}{total\PYZus{}count} \PY{o}{=} \PY{n}{wine}\PY{p}{[}\PY{l+s+s1}{\PYZsq{}}\PY{l+s+s1}{data}\PY{l+s+s1}{\PYZsq{}}\PY{p}{]}\PY{o}{.}\PY{n}{shape}\PY{p}{[}\PY{l+m+mi}{0}\PY{p}{]}
\PY{n+nb}{print}\PY{p}{(}\PY{l+s+s1}{\PYZsq{}}\PY{l+s+s1}{Всего строк: }\PY{l+s+si}{\PYZob{}\PYZcb{}}\PY{l+s+s1}{\PYZsq{}}\PY{o}{.}\PY{n}{format}\PY{p}{(}\PY{n}{total\PYZus{}count}\PY{p}{)}\PY{p}{)}
\end{Verbatim}
\end{tcolorbox}

    \begin{Verbatim}[commandchars=\\\{\}]
Всего строк: 178
    \end{Verbatim}

    Проверим наличие пустых значений в базе данных:

    \begin{tcolorbox}[breakable, size=fbox, boxrule=1pt, pad at break*=1mm,colback=cellbackground, colframe=cellborder]
\prompt{In}{incolor}{36}{\boxspacing}
\begin{Verbatim}[commandchars=\\\{\}]
\PY{k}{for} \PY{n}{col} \PY{o+ow}{in} \PY{n}{data1}\PY{o}{.}\PY{n}{columns}\PY{p}{:}
    \PY{n}{temp\PYZus{}null\PYZus{}count} \PY{o}{=} \PY{n}{data1}\PY{p}{[}\PY{n}{data1}\PY{p}{[}\PY{n}{col}\PY{p}{]}\PY{o}{.}\PY{n}{isnull}\PY{p}{(}\PY{p}{)}\PY{p}{]}\PY{o}{.}\PY{n}{shape}\PY{p}{[}\PY{l+m+mi}{0}\PY{p}{]}
    \PY{n+nb}{print}\PY{p}{(}\PY{l+s+s1}{\PYZsq{}}\PY{l+s+si}{\PYZob{}\PYZcb{}}\PY{l+s+s1}{ \PYZhy{} }\PY{l+s+si}{\PYZob{}\PYZcb{}}\PY{l+s+s1}{\PYZsq{}}\PY{o}{.}\PY{n}{format}\PY{p}{(}\PY{n}{col}\PY{p}{,} \PY{n}{temp\PYZus{}null\PYZus{}count}\PY{p}{)}\PY{p}{)}
\end{Verbatim}
\end{tcolorbox}

    \begin{Verbatim}[commandchars=\\\{\}]
alcohol - 0
malic\_acid - 0
ash - 0
alcalinity\_of\_ash - 0
magnesium - 0
total\_phenols - 0
flavanoids - 0
nonflavanoid\_phenols - 0
proanthocyanins - 0
color\_intensity - 0
hue - 0
od280/od315\_of\_diluted\_wines - 0
proline - 0
target - 0
    \end{Verbatim}

    Рассмотрим сновные статические характеристики набора данных

    \begin{tcolorbox}[breakable, size=fbox, boxrule=1pt, pad at break*=1mm,colback=cellbackground, colframe=cellborder]
\prompt{In}{incolor}{37}{\boxspacing}
\begin{Verbatim}[commandchars=\\\{\}]
\PY{n}{data1}\PY{o}{.}\PY{n}{describe}\PY{p}{(}\PY{p}{)}
\end{Verbatim}
\end{tcolorbox}

            \begin{tcolorbox}[breakable, size=fbox, boxrule=.5pt, pad at break*=1mm, opacityfill=0]
\prompt{Out}{outcolor}{37}{\boxspacing}
\begin{Verbatim}[commandchars=\\\{\}]
          alcohol  malic\_acid         ash  alcalinity\_of\_ash   magnesium  \textbackslash{}
count  178.000000  178.000000  178.000000         178.000000  178.000000
mean    13.000618    2.336348    2.366517          19.494944   99.741573
std      0.811827    1.117146    0.274344           3.339564   14.282484
min     11.030000    0.740000    1.360000          10.600000   70.000000
25\%     12.362500    1.602500    2.210000          17.200000   88.000000
50\%     13.050000    1.865000    2.360000          19.500000   98.000000
75\%     13.677500    3.082500    2.557500          21.500000  107.000000
max     14.830000    5.800000    3.230000          30.000000  162.000000

       total\_phenols  flavanoids  nonflavanoid\_phenols  proanthocyanins  \textbackslash{}
count     178.000000  178.000000            178.000000       178.000000
mean        2.295112    2.029270              0.361854         1.590899
std         0.625851    0.998859              0.124453         0.572359
min         0.980000    0.340000              0.130000         0.410000
25\%         1.742500    1.205000              0.270000         1.250000
50\%         2.355000    2.135000              0.340000         1.555000
75\%         2.800000    2.875000              0.437500         1.950000
max         3.880000    5.080000              0.660000         3.580000

       color\_intensity         hue  od280/od315\_of\_diluted\_wines      proline  \textbackslash{}
count       178.000000  178.000000                    178.000000   178.000000
mean          5.058090    0.957449                      2.611685   746.893258
std           2.318286    0.228572                      0.709990   314.907474
min           1.280000    0.480000                      1.270000   278.000000
25\%           3.220000    0.782500                      1.937500   500.500000
50\%           4.690000    0.965000                      2.780000   673.500000
75\%           6.200000    1.120000                      3.170000   985.000000
max          13.000000    1.710000                      4.000000  1680.000000

           target
count  178.000000
mean     0.938202
std      0.775035
min      0.000000
25\%      0.000000
50\%      1.000000
75\%      2.000000
max      2.000000
\end{Verbatim}
\end{tcolorbox}
        
    Определим значения целевого признака:

    \begin{tcolorbox}[breakable, size=fbox, boxrule=1pt, pad at break*=1mm,colback=cellbackground, colframe=cellborder]
\prompt{In}{incolor}{40}{\boxspacing}
\begin{Verbatim}[commandchars=\\\{\}]
\PY{n}{data1}\PY{p}{[}\PY{l+s+s1}{\PYZsq{}}\PY{l+s+s1}{target}\PY{l+s+s1}{\PYZsq{}}\PY{p}{]}\PY{o}{.}\PY{n}{unique}\PY{p}{(}\PY{p}{)}
\end{Verbatim}
\end{tcolorbox}

            \begin{tcolorbox}[breakable, size=fbox, boxrule=.5pt, pad at break*=1mm, opacityfill=0]
\prompt{Out}{outcolor}{40}{\boxspacing}
\begin{Verbatim}[commandchars=\\\{\}]
array([0., 1., 2.])
\end{Verbatim}
\end{tcolorbox}
        
    Целевой признак содержит только значения 0, 1 и 2.

    Визуальное исследование датасета.

    Оценим данные представленные в датасете при помощи некоторых видов
диаграмм.

    Диаграмма рассеивания

    Здесь мы видим соотношение параметров, отражающих количество золы и
щелочность этой золы.

    \begin{tcolorbox}[breakable, size=fbox, boxrule=1pt, pad at break*=1mm,colback=cellbackground, colframe=cellborder]
\prompt{In}{incolor}{64}{\boxspacing}
\begin{Verbatim}[commandchars=\\\{\}]
\PY{n}{fig}\PY{p}{,} \PY{n}{ax} \PY{o}{=} \PY{n}{plt}\PY{o}{.}\PY{n}{subplots}\PY{p}{(}\PY{n}{figsize}\PY{o}{=}\PY{p}{(}\PY{l+m+mi}{10}\PY{p}{,}\PY{l+m+mi}{10}\PY{p}{)}\PY{p}{)} 
\PY{n}{sns}\PY{o}{.}\PY{n}{scatterplot}\PY{p}{(}\PY{n}{ax}\PY{o}{=}\PY{n}{ax}\PY{p}{,} \PY{n}{x}\PY{o}{=}\PY{l+s+s1}{\PYZsq{}}\PY{l+s+s1}{ash}\PY{l+s+s1}{\PYZsq{}}\PY{p}{,} \PY{n}{y}\PY{o}{=}\PY{l+s+s1}{\PYZsq{}}\PY{l+s+s1}{alcalinity\PYZus{}of\PYZus{}ash}\PY{l+s+s1}{\PYZsq{}}\PY{p}{,} \PY{n}{data}\PY{o}{=}\PY{n}{data1}\PY{p}{)}
\end{Verbatim}
\end{tcolorbox}

            \begin{tcolorbox}[breakable, size=fbox, boxrule=.5pt, pad at break*=1mm, opacityfill=0]
\prompt{Out}{outcolor}{64}{\boxspacing}
\begin{Verbatim}[commandchars=\\\{\}]
<AxesSubplot:xlabel='ash', ylabel='alcalinity\_of\_ash'>
\end{Verbatim}
\end{tcolorbox}
        
    \begin{center}
    \adjustimage{max size={0.9\linewidth}{0.9\paperheight}}{Лабораторная работа №1_files/Лабораторная работа №1_24_1.png}
    \end{center}
    { \hspace*{\fill} \\}
    
    На следующем графике мы можем разделить эти точки по принципу
принадлежности к винам разных культиваторов, то есть по ключевому
признаку.

    \begin{tcolorbox}[breakable, size=fbox, boxrule=1pt, pad at break*=1mm,colback=cellbackground, colframe=cellborder]
\prompt{In}{incolor}{56}{\boxspacing}
\begin{Verbatim}[commandchars=\\\{\}]
\PY{n}{fig}\PY{p}{,} \PY{n}{ax} \PY{o}{=} \PY{n}{plt}\PY{o}{.}\PY{n}{subplots}\PY{p}{(}\PY{n}{figsize}\PY{o}{=}\PY{p}{(}\PY{l+m+mi}{10}\PY{p}{,}\PY{l+m+mi}{10}\PY{p}{)}\PY{p}{)} 
\PY{n}{sns}\PY{o}{.}\PY{n}{scatterplot}\PY{p}{(}\PY{n}{ax}\PY{o}{=}\PY{n}{ax}\PY{p}{,} \PY{n}{x}\PY{o}{=}\PY{l+s+s1}{\PYZsq{}}\PY{l+s+s1}{ash}\PY{l+s+s1}{\PYZsq{}}\PY{p}{,} \PY{n}{y}\PY{o}{=}\PY{l+s+s1}{\PYZsq{}}\PY{l+s+s1}{alcalinity\PYZus{}of\PYZus{}ash}\PY{l+s+s1}{\PYZsq{}}\PY{p}{,} \PY{n}{data}\PY{o}{=}\PY{n}{data1}\PY{p}{,} \PY{n}{hue}\PY{o}{=}\PY{l+s+s1}{\PYZsq{}}\PY{l+s+s1}{target}\PY{l+s+s1}{\PYZsq{}}\PY{p}{)}
\end{Verbatim}
\end{tcolorbox}

            \begin{tcolorbox}[breakable, size=fbox, boxrule=.5pt, pad at break*=1mm, opacityfill=0]
\prompt{Out}{outcolor}{56}{\boxspacing}
\begin{Verbatim}[commandchars=\\\{\}]
<AxesSubplot:xlabel='ash', ylabel='alcalinity\_of\_ash'>
\end{Verbatim}
\end{tcolorbox}
        
    \begin{center}
    \adjustimage{max size={0.9\linewidth}{0.9\paperheight}}{Лабораторная работа №1_files/Лабораторная работа №1_26_1.png}
    \end{center}
    { \hspace*{\fill} \\}
    
    Гистограмма

    Данный вид диаграммы позволяет оценить плотность вероятности
распределения данных.

    \begin{tcolorbox}[breakable, size=fbox, boxrule=1pt, pad at break*=1mm,colback=cellbackground, colframe=cellborder]
\prompt{In}{incolor}{86}{\boxspacing}
\begin{Verbatim}[commandchars=\\\{\}]
\PY{n}{fig}\PY{p}{,} \PY{n}{ax} \PY{o}{=} \PY{n}{plt}\PY{o}{.}\PY{n}{subplots}\PY{p}{(}\PY{n}{figsize}\PY{o}{=}\PY{p}{(}\PY{l+m+mi}{10}\PY{p}{,}\PY{l+m+mi}{10}\PY{p}{)}\PY{p}{)} 
\PY{n}{sns}\PY{o}{.}\PY{n}{histplot}\PY{p}{(}\PY{n}{data1}\PY{p}{[}\PY{l+s+s1}{\PYZsq{}}\PY{l+s+s1}{ash}\PY{l+s+s1}{\PYZsq{}}\PY{p}{]}\PY{p}{)}
\end{Verbatim}
\end{tcolorbox}

            \begin{tcolorbox}[breakable, size=fbox, boxrule=.5pt, pad at break*=1mm, opacityfill=0]
\prompt{Out}{outcolor}{86}{\boxspacing}
\begin{Verbatim}[commandchars=\\\{\}]
<AxesSubplot:xlabel='ash', ylabel='Count'>
\end{Verbatim}
\end{tcolorbox}
        
    \begin{center}
    \adjustimage{max size={0.9\linewidth}{0.9\paperheight}}{Лабораторная работа №1_files/Лабораторная работа №1_29_1.png}
    \end{center}
    { \hspace*{\fill} \\}
    
    Объединяющая диаграмма

    Рассмотрим комбинацию графика и гистрограмм для обоих параметров.

    \begin{tcolorbox}[breakable, size=fbox, boxrule=1pt, pad at break*=1mm,colback=cellbackground, colframe=cellborder]
\prompt{In}{incolor}{78}{\boxspacing}
\begin{Verbatim}[commandchars=\\\{\}]
\PY{n}{sns}\PY{o}{.}\PY{n}{jointplot}\PY{p}{(}\PY{n}{x}\PY{o}{=}\PY{l+s+s1}{\PYZsq{}}\PY{l+s+s1}{ash}\PY{l+s+s1}{\PYZsq{}}\PY{p}{,} \PY{n}{y}\PY{o}{=}\PY{l+s+s1}{\PYZsq{}}\PY{l+s+s1}{alcalinity\PYZus{}of\PYZus{}ash}\PY{l+s+s1}{\PYZsq{}}\PY{p}{,} \PY{n}{data}\PY{o}{=}\PY{n}{data1}\PY{p}{,} \PY{n}{hue}\PY{o}{=}\PY{l+s+s1}{\PYZsq{}}\PY{l+s+s1}{target}\PY{l+s+s1}{\PYZsq{}}\PY{p}{)}
\end{Verbatim}
\end{tcolorbox}

            \begin{tcolorbox}[breakable, size=fbox, boxrule=.5pt, pad at break*=1mm, opacityfill=0]
\prompt{Out}{outcolor}{78}{\boxspacing}
\begin{Verbatim}[commandchars=\\\{\}]
<seaborn.axisgrid.JointGrid at 0x12bfa5700>
\end{Verbatim}
\end{tcolorbox}
        
    \begin{center}
    \adjustimage{max size={0.9\linewidth}{0.9\paperheight}}{Лабораторная работа №1_files/Лабораторная работа №1_32_1.png}
    \end{center}
    { \hspace*{\fill} \\}
    
    \begin{tcolorbox}[breakable, size=fbox, boxrule=1pt, pad at break*=1mm,colback=cellbackground, colframe=cellborder]
\prompt{In}{incolor}{85}{\boxspacing}
\begin{Verbatim}[commandchars=\\\{\}]
\PY{n}{sns}\PY{o}{.}\PY{n}{jointplot}\PY{p}{(}\PY{n}{x}\PY{o}{=}\PY{l+s+s1}{\PYZsq{}}\PY{l+s+s1}{ash}\PY{l+s+s1}{\PYZsq{}}\PY{p}{,} \PY{n}{y}\PY{o}{=}\PY{l+s+s1}{\PYZsq{}}\PY{l+s+s1}{alcalinity\PYZus{}of\PYZus{}ash}\PY{l+s+s1}{\PYZsq{}}\PY{p}{,} \PY{n}{data}\PY{o}{=}\PY{n}{data1}\PY{p}{,} \PY{n}{kind}\PY{o}{=}\PY{l+s+s1}{\PYZsq{}}\PY{l+s+s1}{kde}\PY{l+s+s1}{\PYZsq{}}\PY{p}{)}
\end{Verbatim}
\end{tcolorbox}

            \begin{tcolorbox}[breakable, size=fbox, boxrule=.5pt, pad at break*=1mm, opacityfill=0]
\prompt{Out}{outcolor}{85}{\boxspacing}
\begin{Verbatim}[commandchars=\\\{\}]
<seaborn.axisgrid.JointGrid at 0x12c21c520>
\end{Verbatim}
\end{tcolorbox}
        
    \begin{center}
    \adjustimage{max size={0.9\linewidth}{0.9\paperheight}}{Лабораторная работа №1_files/Лабораторная работа №1_33_1.png}
    \end{center}
    { \hspace*{\fill} \\}
    
    Парные диаграммы

    Функция, отвечающая за ётот вид диаграмм готовит матрицу графиков,
представляющих собой комбинацию диаграмм рассеивания для всего набора
тадных. На главной диагонали при этом строятся гистограммы распределения
соответствующих показателей.

    Рассмотрим графики с учетом распределения по ключевому признаку.

    \begin{tcolorbox}[breakable, size=fbox, boxrule=1pt, pad at break*=1mm,colback=cellbackground, colframe=cellborder]
\prompt{In}{incolor}{92}{\boxspacing}
\begin{Verbatim}[commandchars=\\\{\}]
\PY{n}{sns}\PY{o}{.}\PY{n}{pairplot}\PY{p}{(}\PY{n}{data1}\PY{p}{,} \PY{n}{hue}\PY{o}{=}\PY{l+s+s2}{\PYZdq{}}\PY{l+s+s2}{target}\PY{l+s+s2}{\PYZdq{}}\PY{p}{)}
\end{Verbatim}
\end{tcolorbox}

            \begin{tcolorbox}[breakable, size=fbox, boxrule=.5pt, pad at break*=1mm, opacityfill=0]
\prompt{Out}{outcolor}{92}{\boxspacing}
\begin{Verbatim}[commandchars=\\\{\}]
<seaborn.axisgrid.PairGrid at 0x137e816a0>
\end{Verbatim}
\end{tcolorbox}
        
    \begin{center}
    \adjustimage{max size={0.9\linewidth}{0.9\paperheight}}{Лабораторная работа №1_files/Лабораторная работа №1_37_1.png}
    \end{center}
    { \hspace*{\fill} \\}
    
    Ящик с усами

    Данный тип диаграмм отображает одномерное распределение вероятности.

    По горизонтали:

    \begin{tcolorbox}[breakable, size=fbox, boxrule=1pt, pad at break*=1mm,colback=cellbackground, colframe=cellborder]
\prompt{In}{incolor}{101}{\boxspacing}
\begin{Verbatim}[commandchars=\\\{\}]
\PY{n}{sns}\PY{o}{.}\PY{n}{boxplot}\PY{p}{(}\PY{n}{x}\PY{o}{=}\PY{n}{data1}\PY{p}{[}\PY{l+s+s1}{\PYZsq{}}\PY{l+s+s1}{ash}\PY{l+s+s1}{\PYZsq{}}\PY{p}{]}\PY{p}{)}
\end{Verbatim}
\end{tcolorbox}

            \begin{tcolorbox}[breakable, size=fbox, boxrule=.5pt, pad at break*=1mm, opacityfill=0]
\prompt{Out}{outcolor}{101}{\boxspacing}
\begin{Verbatim}[commandchars=\\\{\}]
<AxesSubplot:xlabel='ash'>
\end{Verbatim}
\end{tcolorbox}
        
    \begin{center}
    \adjustimage{max size={0.9\linewidth}{0.9\paperheight}}{Лабораторная работа №1_files/Лабораторная работа №1_41_1.png}
    \end{center}
    { \hspace*{\fill} \\}
    
    По верткали (с учетом ключегового признака):

    \begin{tcolorbox}[breakable, size=fbox, boxrule=1pt, pad at break*=1mm,colback=cellbackground, colframe=cellborder]
\prompt{In}{incolor}{103}{\boxspacing}
\begin{Verbatim}[commandchars=\\\{\}]
\PY{n}{sns}\PY{o}{.}\PY{n}{boxplot}\PY{p}{(}\PY{n}{x}\PY{o}{=}\PY{l+s+s1}{\PYZsq{}}\PY{l+s+s1}{target}\PY{l+s+s1}{\PYZsq{}}\PY{p}{,} \PY{n}{y}\PY{o}{=}\PY{l+s+s1}{\PYZsq{}}\PY{l+s+s1}{ash}\PY{l+s+s1}{\PYZsq{}}\PY{p}{,} \PY{n}{data}\PY{o}{=}\PY{n}{data1}\PY{p}{)}
\end{Verbatim}
\end{tcolorbox}

            \begin{tcolorbox}[breakable, size=fbox, boxrule=.5pt, pad at break*=1mm, opacityfill=0]
\prompt{Out}{outcolor}{103}{\boxspacing}
\begin{Verbatim}[commandchars=\\\{\}]
<AxesSubplot:xlabel='target', ylabel='ash'>
\end{Verbatim}
\end{tcolorbox}
        
    \begin{center}
    \adjustimage{max size={0.9\linewidth}{0.9\paperheight}}{Лабораторная работа №1_files/Лабораторная работа №1_43_1.png}
    \end{center}
    { \hspace*{\fill} \\}
    
    Скрипкова диаграмма

    Данная диаграмма аналогична предыдущей, но по её краям отображается
распределение плотности.

    \begin{tcolorbox}[breakable, size=fbox, boxrule=1pt, pad at break*=1mm,colback=cellbackground, colframe=cellborder]
\prompt{In}{incolor}{110}{\boxspacing}
\begin{Verbatim}[commandchars=\\\{\}]
\PY{n}{sns}\PY{o}{.}\PY{n}{violinplot}\PY{p}{(}\PY{n}{x}\PY{o}{=}\PY{n}{data1}\PY{p}{[}\PY{l+s+s1}{\PYZsq{}}\PY{l+s+s1}{ash}\PY{l+s+s1}{\PYZsq{}}\PY{p}{]}\PY{p}{)}
\end{Verbatim}
\end{tcolorbox}

            \begin{tcolorbox}[breakable, size=fbox, boxrule=.5pt, pad at break*=1mm, opacityfill=0]
\prompt{Out}{outcolor}{110}{\boxspacing}
\begin{Verbatim}[commandchars=\\\{\}]
<AxesSubplot:xlabel='ash'>
\end{Verbatim}
\end{tcolorbox}
        
    \begin{center}
    \adjustimage{max size={0.9\linewidth}{0.9\paperheight}}{Лабораторная работа №1_files/Лабораторная работа №1_46_1.png}
    \end{center}
    { \hspace*{\fill} \\}
    
    Проверим правильность отображения по ранее исполььзованному графику.

    \begin{tcolorbox}[breakable, size=fbox, boxrule=1pt, pad at break*=1mm,colback=cellbackground, colframe=cellborder]
\prompt{In}{incolor}{114}{\boxspacing}
\begin{Verbatim}[commandchars=\\\{\}]
\PY{n}{fig}\PY{p}{,} \PY{n}{ax} \PY{o}{=} \PY{n}{plt}\PY{o}{.}\PY{n}{subplots}\PY{p}{(}\PY{l+m+mi}{2}\PY{p}{,} \PY{l+m+mi}{1}\PY{p}{,} \PY{n}{figsize}\PY{o}{=}\PY{p}{(}\PY{l+m+mi}{10}\PY{p}{,}\PY{l+m+mi}{10}\PY{p}{)}\PY{p}{)}
\PY{n}{sns}\PY{o}{.}\PY{n}{violinplot}\PY{p}{(}\PY{n}{ax}\PY{o}{=}\PY{n}{ax}\PY{p}{[}\PY{l+m+mi}{0}\PY{p}{]}\PY{p}{,} \PY{n}{x}\PY{o}{=}\PY{n}{data1}\PY{p}{[}\PY{l+s+s1}{\PYZsq{}}\PY{l+s+s1}{ash}\PY{l+s+s1}{\PYZsq{}}\PY{p}{]}\PY{p}{)}
\PY{n}{sns}\PY{o}{.}\PY{n}{histplot}\PY{p}{(}\PY{n}{data1}\PY{p}{[}\PY{l+s+s1}{\PYZsq{}}\PY{l+s+s1}{ash}\PY{l+s+s1}{\PYZsq{}}\PY{p}{]}\PY{p}{,} \PY{n}{kde}\PY{o}{=}\PY{k+kc}{True}\PY{p}{)}
\end{Verbatim}
\end{tcolorbox}

            \begin{tcolorbox}[breakable, size=fbox, boxrule=.5pt, pad at break*=1mm, opacityfill=0]
\prompt{Out}{outcolor}{114}{\boxspacing}
\begin{Verbatim}[commandchars=\\\{\}]
<AxesSubplot:xlabel='ash', ylabel='Count'>
\end{Verbatim}
\end{tcolorbox}
        
    \begin{center}
    \adjustimage{max size={0.9\linewidth}{0.9\paperheight}}{Лабораторная работа №1_files/Лабораторная работа №1_48_1.png}
    \end{center}
    { \hspace*{\fill} \\}
    
    Можно построить аналогичные графики с распределением по ключевому
признаку.

    \begin{tcolorbox}[breakable, size=fbox, boxrule=1pt, pad at break*=1mm,colback=cellbackground, colframe=cellborder]
\prompt{In}{incolor}{113}{\boxspacing}
\begin{Verbatim}[commandchars=\\\{\}]
\PY{n}{sns}\PY{o}{.}\PY{n}{catplot}\PY{p}{(}\PY{n}{y}\PY{o}{=}\PY{l+s+s1}{\PYZsq{}}\PY{l+s+s1}{ash}\PY{l+s+s1}{\PYZsq{}}\PY{p}{,} \PY{n}{x}\PY{o}{=}\PY{l+s+s1}{\PYZsq{}}\PY{l+s+s1}{target}\PY{l+s+s1}{\PYZsq{}}\PY{p}{,} \PY{n}{data}\PY{o}{=}\PY{n}{data1}\PY{p}{,} \PY{n}{kind}\PY{o}{=}\PY{l+s+s2}{\PYZdq{}}\PY{l+s+s2}{violin}\PY{l+s+s2}{\PYZdq{}}\PY{p}{,} \PY{n}{split}\PY{o}{=}\PY{k+kc}{True}\PY{p}{)}
\end{Verbatim}
\end{tcolorbox}

            \begin{tcolorbox}[breakable, size=fbox, boxrule=.5pt, pad at break*=1mm, opacityfill=0]
\prompt{Out}{outcolor}{113}{\boxspacing}
\begin{Verbatim}[commandchars=\\\{\}]
<seaborn.axisgrid.FacetGrid at 0x13d8a7430>
\end{Verbatim}
\end{tcolorbox}
        
    \begin{center}
    \adjustimage{max size={0.9\linewidth}{0.9\paperheight}}{Лабораторная работа №1_files/Лабораторная работа №1_50_1.png}
    \end{center}
    { \hspace*{\fill} \\}
    
    Информация о корреляции признаков.

    Проверка корреляции признаков позволяет решить две задачи:

    \begin{enumerate}
\def\labelenumi{\arabic{enumi}.}
\tightlist
\item
  Определить ценность признака для построения модели машинного обучения.
  То есть корреляцию признака с целевым признаком.
\item
  Понять какие нецелевые признаки линейно зависимы между собой.
\end{enumerate}

    \begin{tcolorbox}[breakable, size=fbox, boxrule=1pt, pad at break*=1mm,colback=cellbackground, colframe=cellborder]
\prompt{In}{incolor}{116}{\boxspacing}
\begin{Verbatim}[commandchars=\\\{\}]
\PY{n}{data1}\PY{o}{.}\PY{n}{corr}\PY{p}{(}\PY{p}{)}
\end{Verbatim}
\end{tcolorbox}

            \begin{tcolorbox}[breakable, size=fbox, boxrule=.5pt, pad at break*=1mm, opacityfill=0]
\prompt{Out}{outcolor}{116}{\boxspacing}
\begin{Verbatim}[commandchars=\\\{\}]
                               alcohol  malic\_acid       ash  \textbackslash{}
alcohol                       1.000000    0.094397  0.211545
malic\_acid                    0.094397    1.000000  0.164045
ash                           0.211545    0.164045  1.000000
alcalinity\_of\_ash            -0.310235    0.288500  0.443367
magnesium                     0.270798   -0.054575  0.286587
total\_phenols                 0.289101   -0.335167  0.128980
flavanoids                    0.236815   -0.411007  0.115077
nonflavanoid\_phenols         -0.155929    0.292977  0.186230
proanthocyanins               0.136698   -0.220746  0.009652
color\_intensity               0.546364    0.248985  0.258887
hue                          -0.071747   -0.561296 -0.074667
od280/od315\_of\_diluted\_wines  0.072343   -0.368710  0.003911
proline                       0.643720   -0.192011  0.223626
target                       -0.328222    0.437776 -0.049643

                              alcalinity\_of\_ash  magnesium  total\_phenols  \textbackslash{}
alcohol                               -0.310235   0.270798       0.289101
malic\_acid                             0.288500  -0.054575      -0.335167
ash                                    0.443367   0.286587       0.128980
alcalinity\_of\_ash                      1.000000  -0.083333      -0.321113
magnesium                             -0.083333   1.000000       0.214401
total\_phenols                         -0.321113   0.214401       1.000000
flavanoids                            -0.351370   0.195784       0.864564
nonflavanoid\_phenols                   0.361922  -0.256294      -0.449935
proanthocyanins                       -0.197327   0.236441       0.612413
color\_intensity                        0.018732   0.199950      -0.055136
hue                                   -0.273955   0.055398       0.433681
od280/od315\_of\_diluted\_wines          -0.276769   0.066004       0.699949
proline                               -0.440597   0.393351       0.498115
target                                 0.517859  -0.209179      -0.719163

                              flavanoids  nonflavanoid\_phenols  \textbackslash{}
alcohol                         0.236815             -0.155929
malic\_acid                     -0.411007              0.292977
ash                             0.115077              0.186230
alcalinity\_of\_ash              -0.351370              0.361922
magnesium                       0.195784             -0.256294
total\_phenols                   0.864564             -0.449935
flavanoids                      1.000000             -0.537900
nonflavanoid\_phenols           -0.537900              1.000000
proanthocyanins                 0.652692             -0.365845
color\_intensity                -0.172379              0.139057
hue                             0.543479             -0.262640
od280/od315\_of\_diluted\_wines    0.787194             -0.503270
proline                         0.494193             -0.311385
target                         -0.847498              0.489109

                              proanthocyanins  color\_intensity       hue  \textbackslash{}
alcohol                              0.136698         0.546364 -0.071747
malic\_acid                          -0.220746         0.248985 -0.561296
ash                                  0.009652         0.258887 -0.074667
alcalinity\_of\_ash                   -0.197327         0.018732 -0.273955
magnesium                            0.236441         0.199950  0.055398
total\_phenols                        0.612413        -0.055136  0.433681
flavanoids                           0.652692        -0.172379  0.543479
nonflavanoid\_phenols                -0.365845         0.139057 -0.262640
proanthocyanins                      1.000000        -0.025250  0.295544
color\_intensity                     -0.025250         1.000000 -0.521813
hue                                  0.295544        -0.521813  1.000000
od280/od315\_of\_diluted\_wines         0.519067        -0.428815  0.565468
proline                              0.330417         0.316100  0.236183
target                              -0.499130         0.265668 -0.617369

                              od280/od315\_of\_diluted\_wines   proline    target
alcohol                                           0.072343  0.643720 -0.328222
malic\_acid                                       -0.368710 -0.192011  0.437776
ash                                               0.003911  0.223626 -0.049643
alcalinity\_of\_ash                                -0.276769 -0.440597  0.517859
magnesium                                         0.066004  0.393351 -0.209179
total\_phenols                                     0.699949  0.498115 -0.719163
flavanoids                                        0.787194  0.494193 -0.847498
nonflavanoid\_phenols                             -0.503270 -0.311385  0.489109
proanthocyanins                                   0.519067  0.330417 -0.499130
color\_intensity                                  -0.428815  0.316100  0.265668
hue                                               0.565468  0.236183 -0.617369
od280/od315\_of\_diluted\_wines                      1.000000  0.312761 -0.788230
proline                                           0.312761  1.000000 -0.633717
target                                           -0.788230 -0.633717  1.000000
\end{Verbatim}
\end{tcolorbox}
        
    Корреляционная матрица содержит коэффициенты корреляции между всеми
парами признаков. При этом очевидно, что матрица симметрична
относительно главной диагонали, а на самой диагонали находятся единицы.

    Проведем анализ полученной корреляционной матрицы:

    \begin{itemize}
\tightlist
\item
  Признаки, которые точно следует оставить: total\_phenols(0.72),
  flavanoids(0.85), od280/od315\_of\_diluted\_wines(0.79)
\item
  Признаки, которые можно оставить: alcohol(0.33), malic\_acid(0.44),
  alcalinity\_of\_ash(0.52), nonflavanoid\_phenols(0.49),
  proanthocyanins(0.50), hue(0.62), proline(0.63)
\item
  Признаки, которые лучше убрать: ash(0.049), magnesium(0.21),
  color\_intensity(0.27)
\end{itemize}

    \begin{itemize}
\tightlist
\item
  Максимальная корреляция среди пар нецелевых признаков наблюдается
  между total\_phenols и flavanoids (0.86). Один из этих признаков
  следует убрать. Лучше для этого подходит total\_phenols, потому что у
  этого признака слабее корреляция с целевым признаком.
\end{itemize}

    Можно использовать различные коэффициенты корреляции при построении
матрицы. Рассмотрим коэффициенты Пирсона (стандартный), Кендалла и
Спирмена.

    \begin{tcolorbox}[breakable, size=fbox, boxrule=1pt, pad at break*=1mm,colback=cellbackground, colframe=cellborder]
\prompt{In}{incolor}{118}{\boxspacing}
\begin{Verbatim}[commandchars=\\\{\}]
\PY{n}{data1}\PY{o}{.}\PY{n}{corr}\PY{p}{(}\PY{n}{method}\PY{o}{=}\PY{l+s+s1}{\PYZsq{}}\PY{l+s+s1}{pearson}\PY{l+s+s1}{\PYZsq{}}\PY{p}{)}
\end{Verbatim}
\end{tcolorbox}

            \begin{tcolorbox}[breakable, size=fbox, boxrule=.5pt, pad at break*=1mm, opacityfill=0]
\prompt{Out}{outcolor}{118}{\boxspacing}
\begin{Verbatim}[commandchars=\\\{\}]
                               alcohol  malic\_acid       ash  \textbackslash{}
alcohol                       1.000000    0.094397  0.211545
malic\_acid                    0.094397    1.000000  0.164045
ash                           0.211545    0.164045  1.000000
alcalinity\_of\_ash            -0.310235    0.288500  0.443367
magnesium                     0.270798   -0.054575  0.286587
total\_phenols                 0.289101   -0.335167  0.128980
flavanoids                    0.236815   -0.411007  0.115077
nonflavanoid\_phenols         -0.155929    0.292977  0.186230
proanthocyanins               0.136698   -0.220746  0.009652
color\_intensity               0.546364    0.248985  0.258887
hue                          -0.071747   -0.561296 -0.074667
od280/od315\_of\_diluted\_wines  0.072343   -0.368710  0.003911
proline                       0.643720   -0.192011  0.223626
target                       -0.328222    0.437776 -0.049643

                              alcalinity\_of\_ash  magnesium  total\_phenols  \textbackslash{}
alcohol                               -0.310235   0.270798       0.289101
malic\_acid                             0.288500  -0.054575      -0.335167
ash                                    0.443367   0.286587       0.128980
alcalinity\_of\_ash                      1.000000  -0.083333      -0.321113
magnesium                             -0.083333   1.000000       0.214401
total\_phenols                         -0.321113   0.214401       1.000000
flavanoids                            -0.351370   0.195784       0.864564
nonflavanoid\_phenols                   0.361922  -0.256294      -0.449935
proanthocyanins                       -0.197327   0.236441       0.612413
color\_intensity                        0.018732   0.199950      -0.055136
hue                                   -0.273955   0.055398       0.433681
od280/od315\_of\_diluted\_wines          -0.276769   0.066004       0.699949
proline                               -0.440597   0.393351       0.498115
target                                 0.517859  -0.209179      -0.719163

                              flavanoids  nonflavanoid\_phenols  \textbackslash{}
alcohol                         0.236815             -0.155929
malic\_acid                     -0.411007              0.292977
ash                             0.115077              0.186230
alcalinity\_of\_ash              -0.351370              0.361922
magnesium                       0.195784             -0.256294
total\_phenols                   0.864564             -0.449935
flavanoids                      1.000000             -0.537900
nonflavanoid\_phenols           -0.537900              1.000000
proanthocyanins                 0.652692             -0.365845
color\_intensity                -0.172379              0.139057
hue                             0.543479             -0.262640
od280/od315\_of\_diluted\_wines    0.787194             -0.503270
proline                         0.494193             -0.311385
target                         -0.847498              0.489109

                              proanthocyanins  color\_intensity       hue  \textbackslash{}
alcohol                              0.136698         0.546364 -0.071747
malic\_acid                          -0.220746         0.248985 -0.561296
ash                                  0.009652         0.258887 -0.074667
alcalinity\_of\_ash                   -0.197327         0.018732 -0.273955
magnesium                            0.236441         0.199950  0.055398
total\_phenols                        0.612413        -0.055136  0.433681
flavanoids                           0.652692        -0.172379  0.543479
nonflavanoid\_phenols                -0.365845         0.139057 -0.262640
proanthocyanins                      1.000000        -0.025250  0.295544
color\_intensity                     -0.025250         1.000000 -0.521813
hue                                  0.295544        -0.521813  1.000000
od280/od315\_of\_diluted\_wines         0.519067        -0.428815  0.565468
proline                              0.330417         0.316100  0.236183
target                              -0.499130         0.265668 -0.617369

                              od280/od315\_of\_diluted\_wines   proline    target
alcohol                                           0.072343  0.643720 -0.328222
malic\_acid                                       -0.368710 -0.192011  0.437776
ash                                               0.003911  0.223626 -0.049643
alcalinity\_of\_ash                                -0.276769 -0.440597  0.517859
magnesium                                         0.066004  0.393351 -0.209179
total\_phenols                                     0.699949  0.498115 -0.719163
flavanoids                                        0.787194  0.494193 -0.847498
nonflavanoid\_phenols                             -0.503270 -0.311385  0.489109
proanthocyanins                                   0.519067  0.330417 -0.499130
color\_intensity                                  -0.428815  0.316100  0.265668
hue                                               0.565468  0.236183 -0.617369
od280/od315\_of\_diluted\_wines                      1.000000  0.312761 -0.788230
proline                                           0.312761  1.000000 -0.633717
target                                           -0.788230 -0.633717  1.000000
\end{Verbatim}
\end{tcolorbox}
        
    \begin{tcolorbox}[breakable, size=fbox, boxrule=1pt, pad at break*=1mm,colback=cellbackground, colframe=cellborder]
\prompt{In}{incolor}{119}{\boxspacing}
\begin{Verbatim}[commandchars=\\\{\}]
\PY{n}{data1}\PY{o}{.}\PY{n}{corr}\PY{p}{(}\PY{n}{method}\PY{o}{=}\PY{l+s+s1}{\PYZsq{}}\PY{l+s+s1}{kendall}\PY{l+s+s1}{\PYZsq{}}\PY{p}{)}
\end{Verbatim}
\end{tcolorbox}

            \begin{tcolorbox}[breakable, size=fbox, boxrule=.5pt, pad at break*=1mm, opacityfill=0]
\prompt{Out}{outcolor}{119}{\boxspacing}
\begin{Verbatim}[commandchars=\\\{\}]
                               alcohol  malic\_acid       ash  \textbackslash{}
alcohol                       1.000000    0.093844  0.170154
malic\_acid                    0.093844    1.000000  0.158178
ash                           0.170154    0.158178  1.000000
alcalinity\_of\_ash            -0.212978    0.210119  0.258352
magnesium                     0.250506    0.050869  0.254246
total\_phenols                 0.209099   -0.174929  0.089855
flavanoids                    0.191087   -0.211918  0.049474
nonflavanoid\_phenols         -0.109554    0.175129  0.098937
proanthocyanins               0.133526   -0.168714  0.018240
color\_intensity               0.434353    0.195607  0.187786
hue                          -0.021717   -0.388707 -0.037234
od280/od315\_of\_diluted\_wines  0.061513   -0.162909 -0.006341
proline                       0.449387   -0.044660  0.171574
target                       -0.238984    0.247494 -0.038085

                              alcalinity\_of\_ash  magnesium  total\_phenols  \textbackslash{}
alcohol                               -0.212978   0.250506       0.209099
malic\_acid                             0.210119   0.050869      -0.174929
ash                                    0.258352   0.254246       0.089855
alcalinity\_of\_ash                      1.000000  -0.121005      -0.256669
magnesium                             -0.121005   1.000000       0.172195
total\_phenols                         -0.256669   0.172195       1.000000
flavanoids                            -0.309865   0.161603       0.701999
nonflavanoid\_phenols                   0.278091  -0.158361      -0.310443
proanthocyanins                       -0.171404   0.117871       0.466517
color\_intensity                       -0.057281   0.241781       0.028264
hue                                   -0.239210   0.023760       0.289210
od280/od315\_of\_diluted\_wines          -0.226253   0.034307       0.478267
proline                               -0.313218   0.343016       0.280203
target                                 0.449402  -0.184992      -0.590404

                              flavanoids  nonflavanoid\_phenols  \textbackslash{}
alcohol                         0.191087             -0.109554
malic\_acid                     -0.211918              0.175129
ash                             0.049474              0.098937
alcalinity\_of\_ash              -0.309865              0.278091
magnesium                       0.161603             -0.158361
total\_phenols                   0.701999             -0.310443
flavanoids                      1.000000             -0.378099
nonflavanoid\_phenols           -0.378099              1.000000
proanthocyanins                 0.534615             -0.269189
color\_intensity                 0.028674              0.036065
hue                             0.354372             -0.179755
od280/od315\_of\_diluted\_wines    0.520448             -0.363787
proline                         0.263661             -0.174108
target                         -0.725255              0.379234

                              proanthocyanins  color\_intensity       hue  \textbackslash{}
alcohol                              0.133526         0.434353 -0.021717
malic\_acid                          -0.168714         0.195607 -0.388707
ash                                  0.018240         0.187786 -0.037234
alcalinity\_of\_ash                   -0.171404        -0.057281 -0.239210
magnesium                            0.117871         0.241781  0.023760
total\_phenols                        0.466517         0.028264  0.289210
flavanoids                           0.534615         0.028674  0.354372
nonflavanoid\_phenols                -0.269189         0.036065 -0.179755
proanthocyanins                      1.000000        -0.014962  0.231071
color\_intensity                     -0.014962         1.000000 -0.291561
hue                                  0.231071        -0.291561  1.000000
od280/od315\_of\_diluted\_wines         0.369104        -0.206046  0.324678
proline                              0.204172         0.316632  0.143508
target                              -0.450225         0.065124 -0.479229

                              od280/od315\_of\_diluted\_wines   proline    target
alcohol                                           0.061513  0.449387 -0.238984
malic\_acid                                       -0.162909 -0.044660  0.247494
ash                                              -0.006341  0.171574 -0.038085
alcalinity\_of\_ash                                -0.226253 -0.313218  0.449402
magnesium                                         0.034307  0.343016 -0.184992
total\_phenols                                     0.478267  0.280203 -0.590404
flavanoids                                        0.520448  0.263661 -0.725255
nonflavanoid\_phenols                             -0.363787 -0.174108  0.379234
proanthocyanins                                   0.369104  0.204172 -0.450225
color\_intensity                                  -0.206046  0.316632  0.065124
hue                                               0.324678  0.143508 -0.479229
od280/od315\_of\_diluted\_wines                      1.000000  0.151559 -0.607572
proline                                           0.151559  1.000000 -0.406260
target                                           -0.607572 -0.406260  1.000000
\end{Verbatim}
\end{tcolorbox}
        
    \begin{tcolorbox}[breakable, size=fbox, boxrule=1pt, pad at break*=1mm,colback=cellbackground, colframe=cellborder]
\prompt{In}{incolor}{121}{\boxspacing}
\begin{Verbatim}[commandchars=\\\{\}]
\PY{n}{data1}\PY{o}{.}\PY{n}{corr}\PY{p}{(}\PY{n}{method}\PY{o}{=}\PY{l+s+s1}{\PYZsq{}}\PY{l+s+s1}{spearman}\PY{l+s+s1}{\PYZsq{}}\PY{p}{)}
\end{Verbatim}
\end{tcolorbox}

            \begin{tcolorbox}[breakable, size=fbox, boxrule=.5pt, pad at break*=1mm, opacityfill=0]
\prompt{Out}{outcolor}{121}{\boxspacing}
\begin{Verbatim}[commandchars=\\\{\}]
                               alcohol  malic\_acid       ash  \textbackslash{}
alcohol                       1.000000    0.140430  0.243722
malic\_acid                    0.140430    1.000000  0.230674
ash                           0.243722    0.230674  1.000000
alcalinity\_of\_ash            -0.306598    0.304069  0.366374
magnesium                     0.365503    0.080188  0.361488
total\_phenols                 0.310920   -0.280225  0.132193
flavanoids                    0.294740   -0.325202  0.078796
nonflavanoid\_phenols         -0.162207    0.255236  0.145583
proanthocyanins               0.192734   -0.244825  0.024384
color\_intensity               0.635425    0.290307  0.283047
hue                          -0.024203   -0.560265 -0.050183
od280/od315\_of\_diluted\_wines  0.103050   -0.255185 -0.007500
proline                       0.633580   -0.057466  0.253163
target                       -0.354167    0.346913 -0.053988

                              alcalinity\_of\_ash  magnesium  total\_phenols  \textbackslash{}
alcohol                               -0.306598   0.365503       0.310920
malic\_acid                             0.304069   0.080188      -0.280225
ash                                    0.366374   0.361488       0.132193
alcalinity\_of\_ash                      1.000000  -0.169558      -0.376657
magnesium                             -0.169558   1.000000       0.246417
total\_phenols                         -0.376657   0.246417       1.000000
flavanoids                            -0.443770   0.233167       0.879404
nonflavanoid\_phenols                   0.389390  -0.236786      -0.448013
proanthocyanins                       -0.253695   0.173647       0.666689
color\_intensity                       -0.073776   0.357029       0.011162
hue                                   -0.352507   0.036095       0.439457
od280/od315\_of\_diluted\_wines          -0.325890   0.056963       0.687207
proline                               -0.456090   0.507575       0.419470
target                                 0.569792  -0.250498      -0.726544

                              flavanoids  nonflavanoid\_phenols  \textbackslash{}
alcohol                         0.294740             -0.162207
malic\_acid                     -0.325202              0.255236
ash                             0.078796              0.145583
alcalinity\_of\_ash              -0.443770              0.389390
magnesium                       0.233167             -0.236786
total\_phenols                   0.879404             -0.448013
flavanoids                      1.000000             -0.543897
nonflavanoid\_phenols           -0.543897              1.000000
proanthocyanins                 0.730322             -0.384629
color\_intensity                -0.042910              0.059639
hue                             0.535430             -0.267813
od280/od315\_of\_diluted\_wines    0.741533             -0.494950
proline                         0.429904             -0.270112
target                         -0.854908              0.474205

                              proanthocyanins  color\_intensity       hue  \textbackslash{}
alcohol                              0.192734         0.635425 -0.024203
malic\_acid                          -0.244825         0.290307 -0.560265
ash                                  0.024384         0.283047 -0.050183
alcalinity\_of\_ash                   -0.253695        -0.073776 -0.352507
magnesium                            0.173647         0.357029  0.036095
total\_phenols                        0.666689         0.011162  0.439457
flavanoids                           0.730322        -0.042910  0.535430
nonflavanoid\_phenols                -0.384629         0.059639 -0.267813
proanthocyanins                      1.000000        -0.030947  0.342795
color\_intensity                     -0.030947         1.000000 -0.418522
hue                                  0.342795        -0.418522  1.000000
od280/od315\_of\_diluted\_wines         0.554031        -0.317516  0.485454
proline                              0.308249         0.457096  0.207740
target                              -0.570648         0.131170 -0.616570

                              od280/od315\_of\_diluted\_wines   proline    target
alcohol                                           0.103050  0.633580 -0.354167
malic\_acid                                       -0.255185 -0.057466  0.346913
ash                                              -0.007500  0.253163 -0.053988
alcalinity\_of\_ash                                -0.325890 -0.456090  0.569792
magnesium                                         0.056963  0.507575 -0.250498
total\_phenols                                     0.687207  0.419470 -0.726544
flavanoids                                        0.741533  0.429904 -0.854908
nonflavanoid\_phenols                             -0.494950 -0.270112  0.474205
proanthocyanins                                   0.554031  0.308249 -0.570648
color\_intensity                                  -0.317516  0.457096  0.131170
hue                                               0.485454  0.207740 -0.616570
od280/od315\_of\_diluted\_wines                      1.000000  0.253266 -0.743787
proline                                           0.253266  1.000000 -0.576383
target                                           -0.743787 -0.576383  1.000000
\end{Verbatim}
\end{tcolorbox}
        
    В использованном датасете довольно большое количество признаков, что
усложняет использование таблиц. Поэтому для удобства модно использовать
тепловые карты, которые так же будут отражать степень корреляции.

    \begin{tcolorbox}[breakable, size=fbox, boxrule=1pt, pad at break*=1mm,colback=cellbackground, colframe=cellborder]
\prompt{In}{incolor}{124}{\boxspacing}
\begin{Verbatim}[commandchars=\\\{\}]
\PY{n}{sns}\PY{o}{.}\PY{n}{heatmap}\PY{p}{(}\PY{n}{data1}\PY{o}{.}\PY{n}{corr}\PY{p}{(}\PY{p}{)}\PY{p}{)}
\end{Verbatim}
\end{tcolorbox}

            \begin{tcolorbox}[breakable, size=fbox, boxrule=.5pt, pad at break*=1mm, opacityfill=0]
\prompt{Out}{outcolor}{124}{\boxspacing}
\begin{Verbatim}[commandchars=\\\{\}]
<AxesSubplot:>
\end{Verbatim}
\end{tcolorbox}
        
    \begin{center}
    \adjustimage{max size={0.9\linewidth}{0.9\paperheight}}{Лабораторная работа №1_files/Лабораторная работа №1_64_1.png}
    \end{center}
    { \hspace*{\fill} \\}
    
    Для ещё большей оптимизации можно оставить лишь одну половину матрицы,
учитывая то, что вторая половина дублирует первую.

    \begin{tcolorbox}[breakable, size=fbox, boxrule=1pt, pad at break*=1mm,colback=cellbackground, colframe=cellborder]
\prompt{In}{incolor}{128}{\boxspacing}
\begin{Verbatim}[commandchars=\\\{\}]
\PY{n}{mask} \PY{o}{=} \PY{n}{np}\PY{o}{.}\PY{n}{zeros\PYZus{}like}\PY{p}{(}\PY{n}{data1}\PY{o}{.}\PY{n}{corr}\PY{p}{(}\PY{p}{)}\PY{p}{,} \PY{n}{dtype}\PY{o}{=}\PY{n+nb}{bool}\PY{p}{)}
\PY{n}{mask}\PY{p}{[}\PY{n}{np}\PY{o}{.}\PY{n}{tril\PYZus{}indices\PYZus{}from}\PY{p}{(}\PY{n}{mask}\PY{p}{)}\PY{p}{]} \PY{o}{=} \PY{k+kc}{True}
\PY{n}{sns}\PY{o}{.}\PY{n}{heatmap}\PY{p}{(}\PY{n}{data1}\PY{o}{.}\PY{n}{corr}\PY{p}{(}\PY{p}{)}\PY{p}{,} \PY{n}{mask}\PY{o}{=}\PY{n}{mask}\PY{p}{)}
\end{Verbatim}
\end{tcolorbox}

            \begin{tcolorbox}[breakable, size=fbox, boxrule=.5pt, pad at break*=1mm, opacityfill=0]
\prompt{Out}{outcolor}{128}{\boxspacing}
\begin{Verbatim}[commandchars=\\\{\}]
<AxesSubplot:>
\end{Verbatim}
\end{tcolorbox}
        
    \begin{center}
    \adjustimage{max size={0.9\linewidth}{0.9\paperheight}}{Лабораторная работа №1_files/Лабораторная работа №1_66_1.png}
    \end{center}
    { \hspace*{\fill} \\}
    
    В таком же виде можно провести сравнение таблиц, построенных по разным
коэффициентам.

    \begin{tcolorbox}[breakable, size=fbox, boxrule=1pt, pad at break*=1mm,colback=cellbackground, colframe=cellborder]
\prompt{In}{incolor}{129}{\boxspacing}
\begin{Verbatim}[commandchars=\\\{\}]
\PY{n}{fig}\PY{p}{,} \PY{n}{ax} \PY{o}{=} \PY{n}{plt}\PY{o}{.}\PY{n}{subplots}\PY{p}{(}\PY{l+m+mi}{1}\PY{p}{,} \PY{l+m+mi}{3}\PY{p}{,} \PY{n}{sharex}\PY{o}{=}\PY{l+s+s1}{\PYZsq{}}\PY{l+s+s1}{col}\PY{l+s+s1}{\PYZsq{}}\PY{p}{,} \PY{n}{sharey}\PY{o}{=}\PY{l+s+s1}{\PYZsq{}}\PY{l+s+s1}{row}\PY{l+s+s1}{\PYZsq{}}\PY{p}{,} \PY{n}{figsize}\PY{o}{=}\PY{p}{(}\PY{l+m+mi}{15}\PY{p}{,}\PY{l+m+mi}{5}\PY{p}{)}\PY{p}{)}
\PY{n}{sns}\PY{o}{.}\PY{n}{heatmap}\PY{p}{(}\PY{n}{data1}\PY{o}{.}\PY{n}{corr}\PY{p}{(}\PY{n}{method}\PY{o}{=}\PY{l+s+s1}{\PYZsq{}}\PY{l+s+s1}{pearson}\PY{l+s+s1}{\PYZsq{}}\PY{p}{)}\PY{p}{,} \PY{n}{ax}\PY{o}{=}\PY{n}{ax}\PY{p}{[}\PY{l+m+mi}{0}\PY{p}{]}\PY{p}{)}
\PY{n}{sns}\PY{o}{.}\PY{n}{heatmap}\PY{p}{(}\PY{n}{data1}\PY{o}{.}\PY{n}{corr}\PY{p}{(}\PY{n}{method}\PY{o}{=}\PY{l+s+s1}{\PYZsq{}}\PY{l+s+s1}{kendall}\PY{l+s+s1}{\PYZsq{}}\PY{p}{)}\PY{p}{,} \PY{n}{ax}\PY{o}{=}\PY{n}{ax}\PY{p}{[}\PY{l+m+mi}{1}\PY{p}{]}\PY{p}{)}
\PY{n}{sns}\PY{o}{.}\PY{n}{heatmap}\PY{p}{(}\PY{n}{data1}\PY{o}{.}\PY{n}{corr}\PY{p}{(}\PY{n}{method}\PY{o}{=}\PY{l+s+s1}{\PYZsq{}}\PY{l+s+s1}{spearman}\PY{l+s+s1}{\PYZsq{}}\PY{p}{)}\PY{p}{,} \PY{n}{ax}\PY{o}{=}\PY{n}{ax}\PY{p}{[}\PY{l+m+mi}{2}\PY{p}{]}\PY{p}{)}
\PY{n}{fig}\PY{o}{.}\PY{n}{suptitle}\PY{p}{(}\PY{l+s+s1}{\PYZsq{}}\PY{l+s+s1}{Корреляционные матрицы, построенные различными методами}\PY{l+s+s1}{\PYZsq{}}\PY{p}{)}
\PY{n}{ax}\PY{p}{[}\PY{l+m+mi}{0}\PY{p}{]}\PY{o}{.}\PY{n}{title}\PY{o}{.}\PY{n}{set\PYZus{}text}\PY{p}{(}\PY{l+s+s1}{\PYZsq{}}\PY{l+s+s1}{Pearson}\PY{l+s+s1}{\PYZsq{}}\PY{p}{)}
\PY{n}{ax}\PY{p}{[}\PY{l+m+mi}{1}\PY{p}{]}\PY{o}{.}\PY{n}{title}\PY{o}{.}\PY{n}{set\PYZus{}text}\PY{p}{(}\PY{l+s+s1}{\PYZsq{}}\PY{l+s+s1}{Kendall}\PY{l+s+s1}{\PYZsq{}}\PY{p}{)}
\PY{n}{ax}\PY{p}{[}\PY{l+m+mi}{2}\PY{p}{]}\PY{o}{.}\PY{n}{title}\PY{o}{.}\PY{n}{set\PYZus{}text}\PY{p}{(}\PY{l+s+s1}{\PYZsq{}}\PY{l+s+s1}{Spearman}\PY{l+s+s1}{\PYZsq{}}\PY{p}{)}
\end{Verbatim}
\end{tcolorbox}

    \begin{center}
    \adjustimage{max size={0.9\linewidth}{0.9\paperheight}}{Лабораторная работа №1_files/Лабораторная работа №1_68_0.png}
    \end{center}
    { \hspace*{\fill} \\}
    

    % Add a bibliography block to the postdoc
    
    
    
\end{document}
