\documentclass[11pt]{article}

    \usepackage[breakable]{tcolorbox}
    \usepackage{parskip} % Stop auto-indenting (to mimic markdown behaviour)
    
    \usepackage{iftex}
        \usepackage[russian,english]{babel}
    \usepackage{fontspec}

    \setsansfont{PT Sans}
    \setmainfont{PT Serif}
    \setmonofont{PT Mono}

    % Basic figure setup, for now with no caption control since it's done
    % automatically by Pandoc (which extracts ![](path) syntax from Markdown).
    \usepackage{graphicx}
    % Maintain compatibility with old templates. Remove in nbconvert 6.0
    \let\Oldincludegraphics\includegraphics
    % Ensure that by default, figures have no caption (until we provide a
    % proper Figure object with a Caption API and a way to capture that
    % in the conversion process - todo).
    \usepackage{caption}
    \DeclareCaptionFormat{nocaption}{}
    \captionsetup{format=nocaption,aboveskip=0pt,belowskip=0pt}

    \usepackage{float}
    \floatplacement{figure}{H} % forces figures to be placed at the correct location
    \usepackage{xcolor} % Allow colors to be defined
    \usepackage{enumerate} % Needed for markdown enumerations to work
    \usepackage{geometry} % Used to adjust the document margins
    \usepackage{amsmath} % Equations
    \usepackage{amssymb} % Equations
    \usepackage{textcomp} % defines textquotesingle
    % Hack from http://tex.stackexchange.com/a/47451/13684:
    \AtBeginDocument{%
        \def\PYZsq{\textquotesingle}% Upright quotes in Pygmentized code
    }
    \usepackage{upquote} % Upright quotes for verbatim code
    \usepackage{eurosym} % defines \euro
    \usepackage[mathletters]{ucs} % Extended unicode (utf-8) support
    \usepackage{fancyvrb} % verbatim replacement that allows latex
    \usepackage{grffile} % extends the file name processing of package graphics 
                         % to support a larger range
    \makeatletter % fix for old versions of grffile with XeLaTeX
    \@ifpackagelater{grffile}{2019/11/01}
    {
      % Do nothing on new versions
    }
    {
      \def\Gread@@xetex#1{%
        \IfFileExists{"\Gin@base".bb}%
        {\Gread@eps{\Gin@base.bb}}%
        {\Gread@@xetex@aux#1}%
      }
    }
    \makeatother
    \usepackage[Export]{adjustbox} % Used to constrain images to a maximum size
    \adjustboxset{max size={0.9\linewidth}{0.9\paperheight}}

    % The hyperref package gives us a pdf with properly built
    % internal navigation ('pdf bookmarks' for the table of contents,
    % internal cross-reference links, web links for URLs, etc.)
    \usepackage{hyperref}
    % The default LaTeX title has an obnoxious amount of whitespace. By default,
    % titling removes some of it. It also provides customization options.
    \usepackage{titling}
    \usepackage{longtable} % longtable support required by pandoc >1.10
    \usepackage{booktabs}  % table support for pandoc > 1.12.2
    \usepackage[inline]{enumitem} % IRkernel/repr support (it uses the enumerate* environment)
    \usepackage[normalem]{ulem} % ulem is needed to support strikethroughs (\sout)
                                % normalem makes italics be italics, not underlines
    \usepackage{mathrsfs}
    

    
    % Colors for the hyperref package
    \definecolor{urlcolor}{rgb}{0,.145,.698}
    \definecolor{linkcolor}{rgb}{.71,0.21,0.01}
    \definecolor{citecolor}{rgb}{.12,.54,.11}

    % ANSI colors
    \definecolor{ansi-black}{HTML}{3E424D}
    \definecolor{ansi-black-intense}{HTML}{282C36}
    \definecolor{ansi-red}{HTML}{E75C58}
    \definecolor{ansi-red-intense}{HTML}{B22B31}
    \definecolor{ansi-green}{HTML}{00A250}
    \definecolor{ansi-green-intense}{HTML}{007427}
    \definecolor{ansi-yellow}{HTML}{DDB62B}
    \definecolor{ansi-yellow-intense}{HTML}{B27D12}
    \definecolor{ansi-blue}{HTML}{208FFB}
    \definecolor{ansi-blue-intense}{HTML}{0065CA}
    \definecolor{ansi-magenta}{HTML}{D160C4}
    \definecolor{ansi-magenta-intense}{HTML}{A03196}
    \definecolor{ansi-cyan}{HTML}{60C6C8}
    \definecolor{ansi-cyan-intense}{HTML}{258F8F}
    \definecolor{ansi-white}{HTML}{C5C1B4}
    \definecolor{ansi-white-intense}{HTML}{A1A6B2}
    \definecolor{ansi-default-inverse-fg}{HTML}{FFFFFF}
    \definecolor{ansi-default-inverse-bg}{HTML}{000000}

    % common color for the border for error outputs.
    \definecolor{outerrorbackground}{HTML}{FFDFDF}

    % commands and environments needed by pandoc snippets
    % extracted from the output of `pandoc -s`
    \providecommand{\tightlist}{%
      \setlength{\itemsep}{0pt}\setlength{\parskip}{0pt}}
    \DefineVerbatimEnvironment{Highlighting}{Verbatim}{commandchars=\\\{\}}
    % Add ',fontsize=\small' for more characters per line
    \newenvironment{Shaded}{}{}
    \newcommand{\KeywordTok}[1]{\textcolor[rgb]{0.00,0.44,0.13}{\textbf{{#1}}}}
    \newcommand{\DataTypeTok}[1]{\textcolor[rgb]{0.56,0.13,0.00}{{#1}}}
    \newcommand{\DecValTok}[1]{\textcolor[rgb]{0.25,0.63,0.44}{{#1}}}
    \newcommand{\BaseNTok}[1]{\textcolor[rgb]{0.25,0.63,0.44}{{#1}}}
    \newcommand{\FloatTok}[1]{\textcolor[rgb]{0.25,0.63,0.44}{{#1}}}
    \newcommand{\CharTok}[1]{\textcolor[rgb]{0.25,0.44,0.63}{{#1}}}
    \newcommand{\StringTok}[1]{\textcolor[rgb]{0.25,0.44,0.63}{{#1}}}
    \newcommand{\CommentTok}[1]{\textcolor[rgb]{0.38,0.63,0.69}{\textit{{#1}}}}
    \newcommand{\OtherTok}[1]{\textcolor[rgb]{0.00,0.44,0.13}{{#1}}}
    \newcommand{\AlertTok}[1]{\textcolor[rgb]{1.00,0.00,0.00}{\textbf{{#1}}}}
    \newcommand{\FunctionTok}[1]{\textcolor[rgb]{0.02,0.16,0.49}{{#1}}}
    \newcommand{\RegionMarkerTok}[1]{{#1}}
    \newcommand{\ErrorTok}[1]{\textcolor[rgb]{1.00,0.00,0.00}{\textbf{{#1}}}}
    \newcommand{\NormalTok}[1]{{#1}}
    
    % Additional commands for more recent versions of Pandoc
    \newcommand{\ConstantTok}[1]{\textcolor[rgb]{0.53,0.00,0.00}{{#1}}}
    \newcommand{\SpecialCharTok}[1]{\textcolor[rgb]{0.25,0.44,0.63}{{#1}}}
    \newcommand{\VerbatimStringTok}[1]{\textcolor[rgb]{0.25,0.44,0.63}{{#1}}}
    \newcommand{\SpecialStringTok}[1]{\textcolor[rgb]{0.73,0.40,0.53}{{#1}}}
    \newcommand{\ImportTok}[1]{{#1}}
    \newcommand{\DocumentationTok}[1]{\textcolor[rgb]{0.73,0.13,0.13}{\textit{{#1}}}}
    \newcommand{\AnnotationTok}[1]{\textcolor[rgb]{0.38,0.63,0.69}{\textbf{\textit{{#1}}}}}
    \newcommand{\CommentVarTok}[1]{\textcolor[rgb]{0.38,0.63,0.69}{\textbf{\textit{{#1}}}}}
    \newcommand{\VariableTok}[1]{\textcolor[rgb]{0.10,0.09,0.49}{{#1}}}
    \newcommand{\ControlFlowTok}[1]{\textcolor[rgb]{0.00,0.44,0.13}{\textbf{{#1}}}}
    \newcommand{\OperatorTok}[1]{\textcolor[rgb]{0.40,0.40,0.40}{{#1}}}
    \newcommand{\BuiltInTok}[1]{{#1}}
    \newcommand{\ExtensionTok}[1]{{#1}}
    \newcommand{\PreprocessorTok}[1]{\textcolor[rgb]{0.74,0.48,0.00}{{#1}}}
    \newcommand{\AttributeTok}[1]{\textcolor[rgb]{0.49,0.56,0.16}{{#1}}}
    \newcommand{\InformationTok}[1]{\textcolor[rgb]{0.38,0.63,0.69}{\textbf{\textit{{#1}}}}}
    \newcommand{\WarningTok}[1]{\textcolor[rgb]{0.38,0.63,0.69}{\textbf{\textit{{#1}}}}}
    
    
    % Define a nice break command that doesn't care if a line doesn't already
    % exist.
    \def\br{\hspace*{\fill} \\* }
    % Math Jax compatibility definitions
    \def\gt{>}
    \def\lt{<}
    \let\Oldtex\TeX
    \let\Oldlatex\LaTeX
    \renewcommand{\TeX}{\textrm{\Oldtex}}
    \renewcommand{\LaTeX}{\textrm{\Oldlatex}}
    % Document parameters
    % Document title
    \title{Лабораторная работа №2}
    
    
    
    
    
% Pygments definitions
\makeatletter
\def\PY@reset{\let\PY@it=\relax \let\PY@bf=\relax%
    \let\PY@ul=\relax \let\PY@tc=\relax%
    \let\PY@bc=\relax \let\PY@ff=\relax}
\def\PY@tok#1{\csname PY@tok@#1\endcsname}
\def\PY@toks#1+{\ifx\relax#1\empty\else%
    \PY@tok{#1}\expandafter\PY@toks\fi}
\def\PY@do#1{\PY@bc{\PY@tc{\PY@ul{%
    \PY@it{\PY@bf{\PY@ff{#1}}}}}}}
\def\PY#1#2{\PY@reset\PY@toks#1+\relax+\PY@do{#2}}

\expandafter\def\csname PY@tok@w\endcsname{\def\PY@tc##1{\textcolor[rgb]{0.73,0.73,0.73}{##1}}}
\expandafter\def\csname PY@tok@c\endcsname{\let\PY@it=\textit\def\PY@tc##1{\textcolor[rgb]{0.25,0.50,0.50}{##1}}}
\expandafter\def\csname PY@tok@cp\endcsname{\def\PY@tc##1{\textcolor[rgb]{0.74,0.48,0.00}{##1}}}
\expandafter\def\csname PY@tok@k\endcsname{\let\PY@bf=\textbf\def\PY@tc##1{\textcolor[rgb]{0.00,0.50,0.00}{##1}}}
\expandafter\def\csname PY@tok@kp\endcsname{\def\PY@tc##1{\textcolor[rgb]{0.00,0.50,0.00}{##1}}}
\expandafter\def\csname PY@tok@kt\endcsname{\def\PY@tc##1{\textcolor[rgb]{0.69,0.00,0.25}{##1}}}
\expandafter\def\csname PY@tok@o\endcsname{\def\PY@tc##1{\textcolor[rgb]{0.40,0.40,0.40}{##1}}}
\expandafter\def\csname PY@tok@ow\endcsname{\let\PY@bf=\textbf\def\PY@tc##1{\textcolor[rgb]{0.67,0.13,1.00}{##1}}}
\expandafter\def\csname PY@tok@nb\endcsname{\def\PY@tc##1{\textcolor[rgb]{0.00,0.50,0.00}{##1}}}
\expandafter\def\csname PY@tok@nf\endcsname{\def\PY@tc##1{\textcolor[rgb]{0.00,0.00,1.00}{##1}}}
\expandafter\def\csname PY@tok@nc\endcsname{\let\PY@bf=\textbf\def\PY@tc##1{\textcolor[rgb]{0.00,0.00,1.00}{##1}}}
\expandafter\def\csname PY@tok@nn\endcsname{\let\PY@bf=\textbf\def\PY@tc##1{\textcolor[rgb]{0.00,0.00,1.00}{##1}}}
\expandafter\def\csname PY@tok@ne\endcsname{\let\PY@bf=\textbf\def\PY@tc##1{\textcolor[rgb]{0.82,0.25,0.23}{##1}}}
\expandafter\def\csname PY@tok@nv\endcsname{\def\PY@tc##1{\textcolor[rgb]{0.10,0.09,0.49}{##1}}}
\expandafter\def\csname PY@tok@no\endcsname{\def\PY@tc##1{\textcolor[rgb]{0.53,0.00,0.00}{##1}}}
\expandafter\def\csname PY@tok@nl\endcsname{\def\PY@tc##1{\textcolor[rgb]{0.63,0.63,0.00}{##1}}}
\expandafter\def\csname PY@tok@ni\endcsname{\let\PY@bf=\textbf\def\PY@tc##1{\textcolor[rgb]{0.60,0.60,0.60}{##1}}}
\expandafter\def\csname PY@tok@na\endcsname{\def\PY@tc##1{\textcolor[rgb]{0.49,0.56,0.16}{##1}}}
\expandafter\def\csname PY@tok@nt\endcsname{\let\PY@bf=\textbf\def\PY@tc##1{\textcolor[rgb]{0.00,0.50,0.00}{##1}}}
\expandafter\def\csname PY@tok@nd\endcsname{\def\PY@tc##1{\textcolor[rgb]{0.67,0.13,1.00}{##1}}}
\expandafter\def\csname PY@tok@s\endcsname{\def\PY@tc##1{\textcolor[rgb]{0.73,0.13,0.13}{##1}}}
\expandafter\def\csname PY@tok@sd\endcsname{\let\PY@it=\textit\def\PY@tc##1{\textcolor[rgb]{0.73,0.13,0.13}{##1}}}
\expandafter\def\csname PY@tok@si\endcsname{\let\PY@bf=\textbf\def\PY@tc##1{\textcolor[rgb]{0.73,0.40,0.53}{##1}}}
\expandafter\def\csname PY@tok@se\endcsname{\let\PY@bf=\textbf\def\PY@tc##1{\textcolor[rgb]{0.73,0.40,0.13}{##1}}}
\expandafter\def\csname PY@tok@sr\endcsname{\def\PY@tc##1{\textcolor[rgb]{0.73,0.40,0.53}{##1}}}
\expandafter\def\csname PY@tok@ss\endcsname{\def\PY@tc##1{\textcolor[rgb]{0.10,0.09,0.49}{##1}}}
\expandafter\def\csname PY@tok@sx\endcsname{\def\PY@tc##1{\textcolor[rgb]{0.00,0.50,0.00}{##1}}}
\expandafter\def\csname PY@tok@m\endcsname{\def\PY@tc##1{\textcolor[rgb]{0.40,0.40,0.40}{##1}}}
\expandafter\def\csname PY@tok@gh\endcsname{\let\PY@bf=\textbf\def\PY@tc##1{\textcolor[rgb]{0.00,0.00,0.50}{##1}}}
\expandafter\def\csname PY@tok@gu\endcsname{\let\PY@bf=\textbf\def\PY@tc##1{\textcolor[rgb]{0.50,0.00,0.50}{##1}}}
\expandafter\def\csname PY@tok@gd\endcsname{\def\PY@tc##1{\textcolor[rgb]{0.63,0.00,0.00}{##1}}}
\expandafter\def\csname PY@tok@gi\endcsname{\def\PY@tc##1{\textcolor[rgb]{0.00,0.63,0.00}{##1}}}
\expandafter\def\csname PY@tok@gr\endcsname{\def\PY@tc##1{\textcolor[rgb]{1.00,0.00,0.00}{##1}}}
\expandafter\def\csname PY@tok@ge\endcsname{\let\PY@it=\textit}
\expandafter\def\csname PY@tok@gs\endcsname{\let\PY@bf=\textbf}
\expandafter\def\csname PY@tok@gp\endcsname{\let\PY@bf=\textbf\def\PY@tc##1{\textcolor[rgb]{0.00,0.00,0.50}{##1}}}
\expandafter\def\csname PY@tok@go\endcsname{\def\PY@tc##1{\textcolor[rgb]{0.53,0.53,0.53}{##1}}}
\expandafter\def\csname PY@tok@gt\endcsname{\def\PY@tc##1{\textcolor[rgb]{0.00,0.27,0.87}{##1}}}
\expandafter\def\csname PY@tok@err\endcsname{\def\PY@bc##1{\setlength{\fboxsep}{0pt}\fcolorbox[rgb]{1.00,0.00,0.00}{1,1,1}{\strut ##1}}}
\expandafter\def\csname PY@tok@kc\endcsname{\let\PY@bf=\textbf\def\PY@tc##1{\textcolor[rgb]{0.00,0.50,0.00}{##1}}}
\expandafter\def\csname PY@tok@kd\endcsname{\let\PY@bf=\textbf\def\PY@tc##1{\textcolor[rgb]{0.00,0.50,0.00}{##1}}}
\expandafter\def\csname PY@tok@kn\endcsname{\let\PY@bf=\textbf\def\PY@tc##1{\textcolor[rgb]{0.00,0.50,0.00}{##1}}}
\expandafter\def\csname PY@tok@kr\endcsname{\let\PY@bf=\textbf\def\PY@tc##1{\textcolor[rgb]{0.00,0.50,0.00}{##1}}}
\expandafter\def\csname PY@tok@bp\endcsname{\def\PY@tc##1{\textcolor[rgb]{0.00,0.50,0.00}{##1}}}
\expandafter\def\csname PY@tok@fm\endcsname{\def\PY@tc##1{\textcolor[rgb]{0.00,0.00,1.00}{##1}}}
\expandafter\def\csname PY@tok@vc\endcsname{\def\PY@tc##1{\textcolor[rgb]{0.10,0.09,0.49}{##1}}}
\expandafter\def\csname PY@tok@vg\endcsname{\def\PY@tc##1{\textcolor[rgb]{0.10,0.09,0.49}{##1}}}
\expandafter\def\csname PY@tok@vi\endcsname{\def\PY@tc##1{\textcolor[rgb]{0.10,0.09,0.49}{##1}}}
\expandafter\def\csname PY@tok@vm\endcsname{\def\PY@tc##1{\textcolor[rgb]{0.10,0.09,0.49}{##1}}}
\expandafter\def\csname PY@tok@sa\endcsname{\def\PY@tc##1{\textcolor[rgb]{0.73,0.13,0.13}{##1}}}
\expandafter\def\csname PY@tok@sb\endcsname{\def\PY@tc##1{\textcolor[rgb]{0.73,0.13,0.13}{##1}}}
\expandafter\def\csname PY@tok@sc\endcsname{\def\PY@tc##1{\textcolor[rgb]{0.73,0.13,0.13}{##1}}}
\expandafter\def\csname PY@tok@dl\endcsname{\def\PY@tc##1{\textcolor[rgb]{0.73,0.13,0.13}{##1}}}
\expandafter\def\csname PY@tok@s2\endcsname{\def\PY@tc##1{\textcolor[rgb]{0.73,0.13,0.13}{##1}}}
\expandafter\def\csname PY@tok@sh\endcsname{\def\PY@tc##1{\textcolor[rgb]{0.73,0.13,0.13}{##1}}}
\expandafter\def\csname PY@tok@s1\endcsname{\def\PY@tc##1{\textcolor[rgb]{0.73,0.13,0.13}{##1}}}
\expandafter\def\csname PY@tok@mb\endcsname{\def\PY@tc##1{\textcolor[rgb]{0.40,0.40,0.40}{##1}}}
\expandafter\def\csname PY@tok@mf\endcsname{\def\PY@tc##1{\textcolor[rgb]{0.40,0.40,0.40}{##1}}}
\expandafter\def\csname PY@tok@mh\endcsname{\def\PY@tc##1{\textcolor[rgb]{0.40,0.40,0.40}{##1}}}
\expandafter\def\csname PY@tok@mi\endcsname{\def\PY@tc##1{\textcolor[rgb]{0.40,0.40,0.40}{##1}}}
\expandafter\def\csname PY@tok@il\endcsname{\def\PY@tc##1{\textcolor[rgb]{0.40,0.40,0.40}{##1}}}
\expandafter\def\csname PY@tok@mo\endcsname{\def\PY@tc##1{\textcolor[rgb]{0.40,0.40,0.40}{##1}}}
\expandafter\def\csname PY@tok@ch\endcsname{\let\PY@it=\textit\def\PY@tc##1{\textcolor[rgb]{0.25,0.50,0.50}{##1}}}
\expandafter\def\csname PY@tok@cm\endcsname{\let\PY@it=\textit\def\PY@tc##1{\textcolor[rgb]{0.25,0.50,0.50}{##1}}}
\expandafter\def\csname PY@tok@cpf\endcsname{\let\PY@it=\textit\def\PY@tc##1{\textcolor[rgb]{0.25,0.50,0.50}{##1}}}
\expandafter\def\csname PY@tok@c1\endcsname{\let\PY@it=\textit\def\PY@tc##1{\textcolor[rgb]{0.25,0.50,0.50}{##1}}}
\expandafter\def\csname PY@tok@cs\endcsname{\let\PY@it=\textit\def\PY@tc##1{\textcolor[rgb]{0.25,0.50,0.50}{##1}}}

\def\PYZbs{\char`\\}
\def\PYZus{\char`\_}
\def\PYZob{\char`\{}
\def\PYZcb{\char`\}}
\def\PYZca{\char`\^}
\def\PYZam{\char`\&}
\def\PYZlt{\char`\<}
\def\PYZgt{\char`\>}
\def\PYZsh{\char`\#}
\def\PYZpc{\char`\%}
\def\PYZdl{\char`\$}
\def\PYZhy{\char`\-}
\def\PYZsq{\char`\'}
\def\PYZdq{\char`\"}
\def\PYZti{\char`\~}
% for compatibility with earlier versions
\def\PYZat{@}
\def\PYZlb{[}
\def\PYZrb{]}
\makeatother


    % For linebreaks inside Verbatim environment from package fancyvrb. 
    \makeatletter
        \newbox\Wrappedcontinuationbox 
        \newbox\Wrappedvisiblespacebox 
        \newcommand*\Wrappedvisiblespace {\textcolor{red}{\textvisiblespace}} 
        \newcommand*\Wrappedcontinuationsymbol {\textcolor{red}{\llap{\tiny$\m@th\hookrightarrow$}}} 
        \newcommand*\Wrappedcontinuationindent {3ex } 
        \newcommand*\Wrappedafterbreak {\kern\Wrappedcontinuationindent\copy\Wrappedcontinuationbox} 
        % Take advantage of the already applied Pygments mark-up to insert 
        % potential linebreaks for TeX processing. 
        %        {, <, #, %, $, ' and ": go to next line. 
        %        _, }, ^, &, >, - and ~: stay at end of broken line. 
        % Use of \textquotesingle for straight quote. 
        \newcommand*\Wrappedbreaksatspecials {% 
            \def\PYGZus{\discretionary{\char`\_}{\Wrappedafterbreak}{\char`\_}}% 
            \def\PYGZob{\discretionary{}{\Wrappedafterbreak\char`\{}{\char`\{}}% 
            \def\PYGZcb{\discretionary{\char`\}}{\Wrappedafterbreak}{\char`\}}}% 
            \def\PYGZca{\discretionary{\char`\^}{\Wrappedafterbreak}{\char`\^}}% 
            \def\PYGZam{\discretionary{\char`\&}{\Wrappedafterbreak}{\char`\&}}% 
            \def\PYGZlt{\discretionary{}{\Wrappedafterbreak\char`\<}{\char`\<}}% 
            \def\PYGZgt{\discretionary{\char`\>}{\Wrappedafterbreak}{\char`\>}}% 
            \def\PYGZsh{\discretionary{}{\Wrappedafterbreak\char`\#}{\char`\#}}% 
            \def\PYGZpc{\discretionary{}{\Wrappedafterbreak\char`\%}{\char`\%}}% 
            \def\PYGZdl{\discretionary{}{\Wrappedafterbreak\char`\$}{\char`\$}}% 
            \def\PYGZhy{\discretionary{\char`\-}{\Wrappedafterbreak}{\char`\-}}% 
            \def\PYGZsq{\discretionary{}{\Wrappedafterbreak\textquotesingle}{\textquotesingle}}% 
            \def\PYGZdq{\discretionary{}{\Wrappedafterbreak\char`\"}{\char`\"}}% 
            \def\PYGZti{\discretionary{\char`\~}{\Wrappedafterbreak}{\char`\~}}% 
        } 
        % Some characters . , ; ? ! / are not pygmentized. 
        % This macro makes them "active" and they will insert potential linebreaks 
        \newcommand*\Wrappedbreaksatpunct {% 
            \lccode`\~`\.\lowercase{\def~}{\discretionary{\hbox{\char`\.}}{\Wrappedafterbreak}{\hbox{\char`\.}}}% 
            \lccode`\~`\,\lowercase{\def~}{\discretionary{\hbox{\char`\,}}{\Wrappedafterbreak}{\hbox{\char`\,}}}% 
            \lccode`\~`\;\lowercase{\def~}{\discretionary{\hbox{\char`\;}}{\Wrappedafterbreak}{\hbox{\char`\;}}}% 
            \lccode`\~`\:\lowercase{\def~}{\discretionary{\hbox{\char`\:}}{\Wrappedafterbreak}{\hbox{\char`\:}}}% 
            \lccode`\~`\?\lowercase{\def~}{\discretionary{\hbox{\char`\?}}{\Wrappedafterbreak}{\hbox{\char`\?}}}% 
            \lccode`\~`\!\lowercase{\def~}{\discretionary{\hbox{\char`\!}}{\Wrappedafterbreak}{\hbox{\char`\!}}}% 
            \lccode`\~`\/\lowercase{\def~}{\discretionary{\hbox{\char`\/}}{\Wrappedafterbreak}{\hbox{\char`\/}}}% 
            \catcode`\.\active
            \catcode`\,\active 
            \catcode`\;\active
            \catcode`\:\active
            \catcode`\?\active
            \catcode`\!\active
            \catcode`\/\active 
            \lccode`\~`\~ 	
        }
    \makeatother

    \let\OriginalVerbatim=\Verbatim
    \makeatletter
    \renewcommand{\Verbatim}[1][1]{%
        %\parskip\z@skip
        \sbox\Wrappedcontinuationbox {\Wrappedcontinuationsymbol}%
        \sbox\Wrappedvisiblespacebox {\FV@SetupFont\Wrappedvisiblespace}%
        \def\FancyVerbFormatLine ##1{\hsize\linewidth
            \vtop{\raggedright\hyphenpenalty\z@\exhyphenpenalty\z@
                \doublehyphendemerits\z@\finalhyphendemerits\z@
                \strut ##1\strut}%
        }%
        % If the linebreak is at a space, the latter will be displayed as visible
        % space at end of first line, and a continuation symbol starts next line.
        % Stretch/shrink are however usually zero for typewriter font.
        \def\FV@Space {%
            \nobreak\hskip\z@ plus\fontdimen3\font minus\fontdimen4\font
            \discretionary{\copy\Wrappedvisiblespacebox}{\Wrappedafterbreak}
            {\kern\fontdimen2\font}%
        }%
        
        % Allow breaks at special characters using \PYG... macros.
        \Wrappedbreaksatspecials
        % Breaks at punctuation characters . , ; ? ! and / need catcode=\active 	
        \OriginalVerbatim[#1,codes*=\Wrappedbreaksatpunct]%
    }
    \makeatother

    % Exact colors from NB
    \definecolor{incolor}{HTML}{303F9F}
    \definecolor{outcolor}{HTML}{D84315}
    \definecolor{cellborder}{HTML}{CFCFCF}
    \definecolor{cellbackground}{HTML}{F7F7F7}
    
    % prompt
    \makeatletter
    \newcommand{\boxspacing}{\kern\kvtcb@left@rule\kern\kvtcb@boxsep}
    \makeatother
    \newcommand{\prompt}[4]{
        {\ttfamily\llap{{\color{#2}[#3]:\hspace{3pt}#4}}\vspace{-\baselineskip}}
    }
    

    
    % Prevent overflowing lines due to hard-to-break entities
    \sloppy 
    % Setup hyperref package
    \hypersetup{
      breaklinks=true,  % so long urls are correctly broken across lines
      colorlinks=true,
      urlcolor=urlcolor,
      linkcolor=linkcolor,
      citecolor=citecolor,
      }
    % Slightly bigger margins than the latex defaults
    
    \geometry{verbose,tmargin=1in,bmargin=1in,lmargin=1in,rmargin=1in}
    
    

\begin{document}
    
    \maketitle
    
    

    
    \hypertarget{ux43eux431ux440ux430ux431ux43eux442ux43aux430-ux43fux440ux43eux43fux443ux441ux43aux43eux432-ux432-ux434ux430ux43dux43dux44bux445-ux43aux43eux434ux438ux440ux43eux432ux430ux43dux438ux435-ux43aux430ux442ux435ux433ux43eux440ux438ux430ux43bux44cux43dux44bux445-ux43fux440ux438ux437ux43dux430ux43aux43eux432-ux43cux430ux441ux448ux442ux430ux431ux438ux440ux43eux432ux430ux43dux438ux435-ux434ux430ux43dux43dux44bux445.}{%
\section{Обработка пропусков в данных, кодирование категориальных
признаков, масштабирование
данных.}\label{ux43eux431ux440ux430ux431ux43eux442ux43aux430-ux43fux440ux43eux43fux443ux441ux43aux43eux432-ux432-ux434ux430ux43dux43dux44bux445-ux43aux43eux434ux438ux440ux43eux432ux430ux43dux438ux435-ux43aux430ux442ux435ux433ux43eux440ux438ux430ux43bux44cux43dux44bux445-ux43fux440ux438ux437ux43dux430ux43aux43eux432-ux43cux430ux441ux448ux442ux430ux431ux438ux440ux43eux432ux430ux43dux438ux435-ux434ux430ux43dux43dux44bux445.}}

\begin{center}\rule{0.5\linewidth}{0.5pt}\end{center}

\hypertarget{ux43eux43fux438ux441ux430ux43dux438ux435-ux437ux430ux434ux430ux43dux438ux44f.}{%
\subsection{1. Описание
задания.}\label{ux43eux43fux438ux441ux430ux43dux438ux435-ux437ux430ux434ux430ux43dux438ux44f.}}

\begin{enumerate}
\def\labelenumi{\arabic{enumi}.}
\tightlist
\item
  Выбрать набор данных (датасет), содержащий категориальные признаки и
  пропуски в данных. Для выполнения следующих пунктов можно использовать
  несколько различных наборов данных (один для обработки пропусков,
  другой для категориальных признаков и т.д.)
\item
  Для выбранного датасета (датасетов) на основе материалов лекции решить
  следующие задачи:

  \begin{itemize}
  \tightlist
  \item
    обработку пропусков в данных;
  \item
    кодирование категориальных признаков;
  \item
    масштабирование данных.
  \end{itemize}
\end{enumerate}

    \hypertarget{ux432ux44bux43fux43eux43bux43dux435ux43dux438ux435-ux440ux430ux431ux43eux442ux44b.}{%
\subsection{2. Выполнение
работы.}\label{ux432ux44bux43fux43eux43bux43dux435ux43dux438ux435-ux440ux430ux431ux43eux442ux44b.}}

    Перед началом работы подключаем необходимые библиотеки:

    \begin{tcolorbox}[breakable, size=fbox, boxrule=1pt, pad at break*=1mm,colback=cellbackground, colframe=cellborder]
\prompt{In}{incolor}{2}{\boxspacing}
\begin{Verbatim}[commandchars=\\\{\}]
\PY{k+kn}{import} \PY{n+nn}{numpy} \PY{k}{as} \PY{n+nn}{np}
\PY{k+kn}{import} \PY{n+nn}{pandas} \PY{k}{as} \PY{n+nn}{pd}
\PY{k+kn}{import} \PY{n+nn}{seaborn} \PY{k}{as} \PY{n+nn}{sns}
\PY{k+kn}{import} \PY{n+nn}{matplotlib}\PY{n+nn}{.}\PY{n+nn}{pyplot} \PY{k}{as} \PY{n+nn}{plt}
\PY{o}{\PYZpc{}}\PY{k}{matplotlib} inline 
\PY{n}{sns}\PY{o}{.}\PY{n}{set}\PY{p}{(}\PY{n}{style}\PY{o}{=}\PY{l+s+s2}{\PYZdq{}}\PY{l+s+s2}{ticks}\PY{l+s+s2}{\PYZdq{}}\PY{p}{)}
\end{Verbatim}
\end{tcolorbox}

    \hypertarget{ux43eux431ux440ux430ux431ux43eux442ux43aux430-ux43fux440ux43eux43fux443ux441ux43aux43eux432-ux434ux430ux43dux43dux44bux445.}{%
\subsubsection{Обработка пропусков
данных.}\label{ux43eux431ux440ux430ux431ux43eux442ux43aux430-ux43fux440ux43eux43fux443ux441ux43aux43eux432-ux434ux430ux43dux43dux44bux445.}}

Для обработки пропуска заднных выберем подходящий датасет.

    \begin{tcolorbox}[breakable, size=fbox, boxrule=1pt, pad at break*=1mm,colback=cellbackground, colframe=cellborder]
\prompt{In}{incolor}{3}{\boxspacing}
\begin{Verbatim}[commandchars=\\\{\}]
\PY{n}{data} \PY{o}{=} \PY{n}{pd}\PY{o}{.}\PY{n}{read\PYZus{}csv}\PY{p}{(}\PY{l+s+s1}{\PYZsq{}}\PY{l+s+s1}{country\PYZus{}vaccinations.csv}\PY{l+s+s1}{\PYZsq{}}\PY{p}{,} \PY{n}{sep}\PY{o}{=}\PY{l+s+s2}{\PYZdq{}}\PY{l+s+s2}{,}\PY{l+s+s2}{\PYZdq{}}\PY{p}{)}
\end{Verbatim}
\end{tcolorbox}

    Данный датасет содержит данные о прогрессе вакцинации в разных странах.

    \begin{tcolorbox}[breakable, size=fbox, boxrule=1pt, pad at break*=1mm,colback=cellbackground, colframe=cellborder]
\prompt{In}{incolor}{66}{\boxspacing}
\begin{Verbatim}[commandchars=\\\{\}]
\PY{n}{data}\PY{o}{.}\PY{n}{shape}
\end{Verbatim}
\end{tcolorbox}

            \begin{tcolorbox}[breakable, size=fbox, boxrule=.5pt, pad at break*=1mm, opacityfill=0]
\prompt{Out}{outcolor}{66}{\boxspacing}
\begin{Verbatim}[commandchars=\\\{\}]
(11156, 15)
\end{Verbatim}
\end{tcolorbox}
        
    Проверим, есть ли пропуски, которые мы могли бы устранить:

    \begin{tcolorbox}[breakable, size=fbox, boxrule=1pt, pad at break*=1mm,colback=cellbackground, colframe=cellborder]
\prompt{In}{incolor}{18}{\boxspacing}
\begin{Verbatim}[commandchars=\\\{\}]
\PY{n}{data}\PY{o}{.}\PY{n}{isnull}\PY{p}{(}\PY{p}{)}\PY{o}{.}\PY{n}{sum}\PY{p}{(}\PY{p}{)}
\end{Verbatim}
\end{tcolorbox}

            \begin{tcolorbox}[breakable, size=fbox, boxrule=.5pt, pad at break*=1mm, opacityfill=0]
\prompt{Out}{outcolor}{18}{\boxspacing}
\begin{Verbatim}[commandchars=\\\{\}]
country                                   0
iso\_code                                  0
date                                      0
total\_vaccinations                     4510
people\_vaccinated                      5169
people\_fully\_vaccinated                6872
daily\_vaccinations\_raw                 5590
daily\_vaccinations                      196
total\_vaccinations\_per\_hundred         4510
people\_vaccinated\_per\_hundred          5169
people\_fully\_vaccinated\_per\_hundred    6872
daily\_vaccinations\_per\_million          196
vaccines                                  0
source\_name                               0
source\_website                            0
dtype: int64
\end{Verbatim}
\end{tcolorbox}
        
    \begin{tcolorbox}[breakable, size=fbox, boxrule=1pt, pad at break*=1mm,colback=cellbackground, colframe=cellborder]
\prompt{In}{incolor}{22}{\boxspacing}
\begin{Verbatim}[commandchars=\\\{\}]
\PY{n}{data}\PY{o}{.}\PY{n}{head}\PY{p}{(}\PY{p}{)}
\end{Verbatim}
\end{tcolorbox}

            \begin{tcolorbox}[breakable, size=fbox, boxrule=.5pt, pad at break*=1mm, opacityfill=0]
\prompt{Out}{outcolor}{22}{\boxspacing}
\begin{Verbatim}[commandchars=\\\{\}]
       country iso\_code        date  total\_vaccinations  people\_vaccinated  \textbackslash{}
0  Afghanistan      AFG  2021-02-22                 0.0                0.0
1  Afghanistan      AFG  2021-02-23                 NaN                NaN
2  Afghanistan      AFG  2021-02-24                 NaN                NaN
3  Afghanistan      AFG  2021-02-25                 NaN                NaN
4  Afghanistan      AFG  2021-02-26                 NaN                NaN

   people\_fully\_vaccinated  daily\_vaccinations\_raw  daily\_vaccinations  \textbackslash{}
0                      NaN                     NaN                 NaN
1                      NaN                     NaN              1367.0
2                      NaN                     NaN              1367.0
3                      NaN                     NaN              1367.0
4                      NaN                     NaN              1367.0

   total\_vaccinations\_per\_hundred  people\_vaccinated\_per\_hundred  \textbackslash{}
0                             0.0                            0.0
1                             NaN                            NaN
2                             NaN                            NaN
3                             NaN                            NaN
4                             NaN                            NaN

   people\_fully\_vaccinated\_per\_hundred  daily\_vaccinations\_per\_million  \textbackslash{}
0                                  NaN                             NaN
1                                  NaN                            35.0
2                                  NaN                            35.0
3                                  NaN                            35.0
4                                  NaN                            35.0

             vaccines                source\_name  \textbackslash{}
0  Oxford/AstraZeneca  Government of Afghanistan
1  Oxford/AstraZeneca  Government of Afghanistan
2  Oxford/AstraZeneca  Government of Afghanistan
3  Oxford/AstraZeneca  Government of Afghanistan
4  Oxford/AstraZeneca  Government of Afghanistan

                                      source\_website
0  http://www.xinhuanet.com/english/asiapacific/2{\ldots}
1  http://www.xinhuanet.com/english/asiapacific/2{\ldots}
2  http://www.xinhuanet.com/english/asiapacific/2{\ldots}
3  http://www.xinhuanet.com/english/asiapacific/2{\ldots}
4  http://www.xinhuanet.com/english/asiapacific/2{\ldots}
\end{Verbatim}
\end{tcolorbox}
        
    Судя по полученным сведениям, датасет имеет множество пропусков, которые
необходимо устранить.

    \hypertarget{ux443ux434ux430ux43bux435ux43dux438ux435-ux438-ux437ux430ux43fux43eux43bux43dux435ux43dux438ux435-ux43dux443ux43bux44fux43cux438.}{%
\paragraph{Удаление и заполнение
нулями.}\label{ux443ux434ux430ux43bux435ux43dux438ux435-ux438-ux437ux430ux43fux43eux43bux43dux435ux43dux438ux435-ux43dux443ux43bux44fux43cux438.}}

    \begin{tcolorbox}[breakable, size=fbox, boxrule=1pt, pad at break*=1mm,colback=cellbackground, colframe=cellborder]
\prompt{In}{incolor}{23}{\boxspacing}
\begin{Verbatim}[commandchars=\\\{\}]
\PY{c+c1}{\PYZsh{} Удаление колонок, содержащих пустые значения}
\PY{n}{data\PYZus{}del\PYZus{}1} \PY{o}{=} \PY{n}{data}\PY{o}{.}\PY{n}{dropna}\PY{p}{(}\PY{n}{axis}\PY{o}{=}\PY{l+m+mi}{1}\PY{p}{,} \PY{n}{how}\PY{o}{=}\PY{l+s+s1}{\PYZsq{}}\PY{l+s+s1}{any}\PY{l+s+s1}{\PYZsq{}}\PY{p}{)}
\PY{p}{(}\PY{n}{data}\PY{o}{.}\PY{n}{shape}\PY{p}{,} \PY{n}{data\PYZus{}del\PYZus{}1}\PY{o}{.}\PY{n}{shape}\PY{p}{)}
\end{Verbatim}
\end{tcolorbox}

            \begin{tcolorbox}[breakable, size=fbox, boxrule=.5pt, pad at break*=1mm, opacityfill=0]
\prompt{Out}{outcolor}{23}{\boxspacing}
\begin{Verbatim}[commandchars=\\\{\}]
((11156, 15), (11156, 6))
\end{Verbatim}
\end{tcolorbox}
        
    \begin{tcolorbox}[breakable, size=fbox, boxrule=1pt, pad at break*=1mm,colback=cellbackground, colframe=cellborder]
\prompt{In}{incolor}{24}{\boxspacing}
\begin{Verbatim}[commandchars=\\\{\}]
\PY{c+c1}{\PYZsh{} Удаление строк, содержащих пустые значения}
\PY{n}{data\PYZus{}del\PYZus{}2} \PY{o}{=} \PY{n}{data}\PY{o}{.}\PY{n}{dropna}\PY{p}{(}\PY{n}{axis}\PY{o}{=}\PY{l+m+mi}{0}\PY{p}{,} \PY{n}{how}\PY{o}{=}\PY{l+s+s1}{\PYZsq{}}\PY{l+s+s1}{any}\PY{l+s+s1}{\PYZsq{}}\PY{p}{)}
\PY{p}{(}\PY{n}{data}\PY{o}{.}\PY{n}{shape}\PY{p}{,} \PY{n}{data\PYZus{}del\PYZus{}2}\PY{o}{.}\PY{n}{shape}\PY{p}{)}
\end{Verbatim}
\end{tcolorbox}

            \begin{tcolorbox}[breakable, size=fbox, boxrule=.5pt, pad at break*=1mm, opacityfill=0]
\prompt{Out}{outcolor}{24}{\boxspacing}
\begin{Verbatim}[commandchars=\\\{\}]
((11156, 15), (3837, 15))
\end{Verbatim}
\end{tcolorbox}
        
    \begin{tcolorbox}[breakable, size=fbox, boxrule=1pt, pad at break*=1mm,colback=cellbackground, colframe=cellborder]
\prompt{In}{incolor}{25}{\boxspacing}
\begin{Verbatim}[commandchars=\\\{\}]
\PY{c+c1}{\PYZsh{} Заполнение всех пропущенных значений нулями}
\PY{n}{data\PYZus{}new\PYZus{}3} \PY{o}{=} \PY{n}{data}\PY{o}{.}\PY{n}{fillna}\PY{p}{(}\PY{l+m+mi}{0}\PY{p}{)}
\PY{n}{data\PYZus{}new\PYZus{}3}\PY{o}{.}\PY{n}{head}\PY{p}{(}\PY{p}{)}
\end{Verbatim}
\end{tcolorbox}

            \begin{tcolorbox}[breakable, size=fbox, boxrule=.5pt, pad at break*=1mm, opacityfill=0]
\prompt{Out}{outcolor}{25}{\boxspacing}
\begin{Verbatim}[commandchars=\\\{\}]
       country iso\_code        date  total\_vaccinations  people\_vaccinated  \textbackslash{}
0  Afghanistan      AFG  2021-02-22                 0.0                0.0
1  Afghanistan      AFG  2021-02-23                 0.0                0.0
2  Afghanistan      AFG  2021-02-24                 0.0                0.0
3  Afghanistan      AFG  2021-02-25                 0.0                0.0
4  Afghanistan      AFG  2021-02-26                 0.0                0.0

   people\_fully\_vaccinated  daily\_vaccinations\_raw  daily\_vaccinations  \textbackslash{}
0                      0.0                     0.0                 0.0
1                      0.0                     0.0              1367.0
2                      0.0                     0.0              1367.0
3                      0.0                     0.0              1367.0
4                      0.0                     0.0              1367.0

   total\_vaccinations\_per\_hundred  people\_vaccinated\_per\_hundred  \textbackslash{}
0                             0.0                            0.0
1                             0.0                            0.0
2                             0.0                            0.0
3                             0.0                            0.0
4                             0.0                            0.0

   people\_fully\_vaccinated\_per\_hundred  daily\_vaccinations\_per\_million  \textbackslash{}
0                                  0.0                             0.0
1                                  0.0                            35.0
2                                  0.0                            35.0
3                                  0.0                            35.0
4                                  0.0                            35.0

             vaccines                source\_name  \textbackslash{}
0  Oxford/AstraZeneca  Government of Afghanistan
1  Oxford/AstraZeneca  Government of Afghanistan
2  Oxford/AstraZeneca  Government of Afghanistan
3  Oxford/AstraZeneca  Government of Afghanistan
4  Oxford/AstraZeneca  Government of Afghanistan

                                      source\_website
0  http://www.xinhuanet.com/english/asiapacific/2{\ldots}
1  http://www.xinhuanet.com/english/asiapacific/2{\ldots}
2  http://www.xinhuanet.com/english/asiapacific/2{\ldots}
3  http://www.xinhuanet.com/english/asiapacific/2{\ldots}
4  http://www.xinhuanet.com/english/asiapacific/2{\ldots}
\end{Verbatim}
\end{tcolorbox}
        
    \hypertarget{ux432ux43dux435ux434ux440ux435ux43dux438ux435-ux437ux43dux430ux447ux435ux43dux438ux439---ux438ux43cux43fux44cux44eux442ux430ux446ux438ux44f}{%
\paragraph{Внедрение значений -
импьютация}\label{ux432ux43dux435ux434ux440ux435ux43dux438ux435-ux437ux43dux430ux447ux435ux43dux438ux439---ux438ux43cux43fux44cux44eux442ux430ux446ux438ux44f}}

Все колонки числовые, поэтому просто составим их список и определим
количество пустых значений:

    \begin{tcolorbox}[breakable, size=fbox, boxrule=1pt, pad at break*=1mm,colback=cellbackground, colframe=cellborder]
\prompt{In}{incolor}{29}{\boxspacing}
\begin{Verbatim}[commandchars=\\\{\}]
\PY{n}{nul\PYZus{}cols} \PY{o}{=} \PY{p}{[}\PY{l+s+s1}{\PYZsq{}}\PY{l+s+s1}{total\PYZus{}vaccinations}\PY{l+s+s1}{\PYZsq{}}\PY{p}{,} \PY{l+s+s1}{\PYZsq{}}\PY{l+s+s1}{people\PYZus{}vaccinated}\PY{l+s+s1}{\PYZsq{}}\PY{p}{,} \PY{l+s+s1}{\PYZsq{}}\PY{l+s+s1}{people\PYZus{}fully\PYZus{}vaccinated}\PY{l+s+s1}{\PYZsq{}}\PY{p}{,} \PY{l+s+s1}{\PYZsq{}}\PY{l+s+s1}{daily\PYZus{}vaccinations\PYZus{}raw}\PY{l+s+s1}{\PYZsq{}}\PY{p}{,} \PY{l+s+s1}{\PYZsq{}}\PY{l+s+s1}{daily\PYZus{}vaccinations}\PY{l+s+s1}{\PYZsq{}}\PY{p}{,} \PY{l+s+s1}{\PYZsq{}}\PY{l+s+s1}{total\PYZus{}vaccinations\PYZus{}per\PYZus{}hundred}\PY{l+s+s1}{\PYZsq{}}\PY{p}{,} \PY{l+s+s1}{\PYZsq{}}\PY{l+s+s1}{people\PYZus{}vaccinated\PYZus{}per\PYZus{}hundred}\PY{l+s+s1}{\PYZsq{}}\PY{p}{]}
\PY{n}{total\PYZus{}count} \PY{o}{=} \PY{n}{data}\PY{o}{.}\PY{n}{shape}\PY{p}{[}\PY{l+m+mi}{0}\PY{p}{]}
\PY{k}{for} \PY{n}{col} \PY{o+ow}{in} \PY{n}{nul\PYZus{}cols}\PY{p}{:}
    \PY{c+c1}{\PYZsh{} Количество пустых значений }
    \PY{n}{temp\PYZus{}null\PYZus{}count} \PY{o}{=} \PY{n}{data}\PY{p}{[}\PY{n}{data}\PY{p}{[}\PY{n}{col}\PY{p}{]}\PY{o}{.}\PY{n}{isnull}\PY{p}{(}\PY{p}{)}\PY{p}{]}\PY{o}{.}\PY{n}{shape}\PY{p}{[}\PY{l+m+mi}{0}\PY{p}{]}
    \PY{n}{dt} \PY{o}{=} \PY{n+nb}{str}\PY{p}{(}\PY{n}{data}\PY{p}{[}\PY{n}{col}\PY{p}{]}\PY{o}{.}\PY{n}{dtype}\PY{p}{)}
    \PY{k}{if} \PY{n}{temp\PYZus{}null\PYZus{}count}\PY{o}{\PYZgt{}}\PY{l+m+mi}{0}\PY{p}{:}
        \PY{n}{temp\PYZus{}perc} \PY{o}{=} \PY{n+nb}{round}\PY{p}{(}\PY{p}{(}\PY{n}{temp\PYZus{}null\PYZus{}count} \PY{o}{/} \PY{n}{total\PYZus{}count}\PY{p}{)} \PY{o}{*} \PY{l+m+mf}{100.0}\PY{p}{,} \PY{l+m+mi}{2}\PY{p}{)}
        \PY{n+nb}{print}\PY{p}{(}\PY{l+s+s1}{\PYZsq{}}\PY{l+s+s1}{Колонка }\PY{l+s+si}{\PYZob{}\PYZcb{}}\PY{l+s+s1}{. Тип данных }\PY{l+s+si}{\PYZob{}\PYZcb{}}\PY{l+s+s1}{. Количество пустых значений }\PY{l+s+si}{\PYZob{}\PYZcb{}}\PY{l+s+s1}{, }\PY{l+s+si}{\PYZob{}\PYZcb{}}\PY{l+s+s1}{\PYZpc{}}\PY{l+s+s1}{.}\PY{l+s+s1}{\PYZsq{}}\PY{o}{.}\PY{n}{format}\PY{p}{(}\PY{n}{col}\PY{p}{,} \PY{n}{dt}\PY{p}{,} \PY{n}{temp\PYZus{}null\PYZus{}count}\PY{p}{,} \PY{n}{temp\PYZus{}perc}\PY{p}{)}\PY{p}{)}
\end{Verbatim}
\end{tcolorbox}

    \begin{Verbatim}[commandchars=\\\{\}]
Колонка total\_vaccinations. Тип данных float64. Количество пустых значений 4510,
40.43\%.
Колонка people\_vaccinated. Тип данных float64. Количество пустых значений 5169,
46.33\%.
Колонка people\_fully\_vaccinated. Тип данных float64. Количество пустых значений
6872, 61.6\%.
Колонка daily\_vaccinations\_raw. Тип данных float64. Количество пустых значений
5590, 50.11\%.
Колонка daily\_vaccinations. Тип данных float64. Количество пустых значений 196,
1.76\%.
Колонка total\_vaccinations\_per\_hundred. Тип данных float64. Количество пустых
значений 4510, 40.43\%.
Колонка people\_vaccinated\_per\_hundred. Тип данных float64. Количество пустых
значений 5169, 46.33\%.
    \end{Verbatim}

    \begin{tcolorbox}[breakable, size=fbox, boxrule=1pt, pad at break*=1mm,colback=cellbackground, colframe=cellborder]
\prompt{In}{incolor}{32}{\boxspacing}
\begin{Verbatim}[commandchars=\\\{\}]
\PY{c+c1}{\PYZsh{} Фильтр по колонкам с пропущенными значениями}
\PY{n}{data\PYZus{}nul} \PY{o}{=} \PY{n}{data}\PY{p}{[}\PY{n}{nul\PYZus{}cols}\PY{p}{]}
\PY{n}{data\PYZus{}nul}
\end{Verbatim}
\end{tcolorbox}

            \begin{tcolorbox}[breakable, size=fbox, boxrule=.5pt, pad at break*=1mm, opacityfill=0]
\prompt{Out}{outcolor}{32}{\boxspacing}
\begin{Verbatim}[commandchars=\\\{\}]
       total\_vaccinations  people\_vaccinated  people\_fully\_vaccinated  \textbackslash{}
0                     0.0                0.0                      NaN
1                     NaN                NaN                      NaN
2                     NaN                NaN                      NaN
3                     NaN                NaN                      NaN
4                     NaN                NaN                      NaN
{\ldots}                   {\ldots}                {\ldots}                      {\ldots}
11151            162633.0           139133.0                  23500.0
11152            179417.0           153238.0                  26179.0
11153            193677.0           166543.0                  27134.0
11154            206205.0           178237.0                  27968.0
11155            222733.0           193936.0                  28797.0

       daily\_vaccinations\_raw  daily\_vaccinations  \textbackslash{}
0                         NaN                 NaN
1                         NaN              1367.0
2                         NaN              1367.0
3                         NaN              1367.0
4                         NaN              1367.0
{\ldots}                       {\ldots}                 {\ldots}
11151                 17123.0             10967.0
11152                 16784.0             12505.0
11153                 14260.0             12624.0
11154                 12528.0             11636.0
11155                 16528.0             12831.0

       total\_vaccinations\_per\_hundred  people\_vaccinated\_per\_hundred
0                                0.00                           0.00
1                                 NaN                            NaN
2                                 NaN                            NaN
3                                 NaN                            NaN
4                                 NaN                            NaN
{\ldots}                               {\ldots}                            {\ldots}
11151                            1.09                           0.94
11152                            1.21                           1.03
11153                            1.30                           1.12
11154                            1.39                           1.20
11155                            1.50                           1.30

[11156 rows x 7 columns]
\end{Verbatim}
\end{tcolorbox}
        
    \begin{tcolorbox}[breakable, size=fbox, boxrule=1pt, pad at break*=1mm,colback=cellbackground, colframe=cellborder]
\prompt{In}{incolor}{33}{\boxspacing}
\begin{Verbatim}[commandchars=\\\{\}]
\PY{c+c1}{\PYZsh{} Гистограмма по признакам}
\PY{k}{for} \PY{n}{col} \PY{o+ow}{in} \PY{n}{data\PYZus{}nul}\PY{p}{:}
    \PY{n}{plt}\PY{o}{.}\PY{n}{hist}\PY{p}{(}\PY{n}{data}\PY{p}{[}\PY{n}{col}\PY{p}{]}\PY{p}{,} \PY{l+m+mi}{50}\PY{p}{)}
    \PY{n}{plt}\PY{o}{.}\PY{n}{xlabel}\PY{p}{(}\PY{n}{col}\PY{p}{)}
    \PY{n}{plt}\PY{o}{.}\PY{n}{show}\PY{p}{(}\PY{p}{)}
\end{Verbatim}
\end{tcolorbox}

    \begin{center}
    \adjustimage{max size={0.9\linewidth}{0.9\paperheight}}{Лабораторная работа №2_files/Лабораторная работа №2_19_0.png}
    \end{center}
    { \hspace*{\fill} \\}
    
    \begin{center}
    \adjustimage{max size={0.9\linewidth}{0.9\paperheight}}{Лабораторная работа №2_files/Лабораторная работа №2_19_1.png}
    \end{center}
    { \hspace*{\fill} \\}
    
    \begin{center}
    \adjustimage{max size={0.9\linewidth}{0.9\paperheight}}{Лабораторная работа №2_files/Лабораторная работа №2_19_2.png}
    \end{center}
    { \hspace*{\fill} \\}
    
    \begin{center}
    \adjustimage{max size={0.9\linewidth}{0.9\paperheight}}{Лабораторная работа №2_files/Лабораторная работа №2_19_3.png}
    \end{center}
    { \hspace*{\fill} \\}
    
    \begin{center}
    \adjustimage{max size={0.9\linewidth}{0.9\paperheight}}{Лабораторная работа №2_files/Лабораторная работа №2_19_4.png}
    \end{center}
    { \hspace*{\fill} \\}
    
    \begin{center}
    \adjustimage{max size={0.9\linewidth}{0.9\paperheight}}{Лабораторная работа №2_files/Лабораторная работа №2_19_5.png}
    \end{center}
    { \hspace*{\fill} \\}
    
    \begin{center}
    \adjustimage{max size={0.9\linewidth}{0.9\paperheight}}{Лабораторная работа №2_files/Лабораторная работа №2_19_6.png}
    \end{center}
    { \hspace*{\fill} \\}
    
    Можно использовать встроенные средства импьютации библиотеки
scikit-learn:

    \begin{tcolorbox}[breakable, size=fbox, boxrule=1pt, pad at break*=1mm,colback=cellbackground, colframe=cellborder]
\prompt{In}{incolor}{37}{\boxspacing}
\begin{Verbatim}[commandchars=\\\{\}]
\PY{k+kn}{from} \PY{n+nn}{sklearn}\PY{n+nn}{.}\PY{n+nn}{impute} \PY{k+kn}{import} \PY{n}{SimpleImputer}
\PY{k+kn}{from} \PY{n+nn}{sklearn}\PY{n+nn}{.}\PY{n+nn}{impute} \PY{k+kn}{import} \PY{n}{MissingIndicator}
\end{Verbatim}
\end{tcolorbox}

    Выберем один из столбцов:

    \begin{tcolorbox}[breakable, size=fbox, boxrule=1pt, pad at break*=1mm,colback=cellbackground, colframe=cellborder]
\prompt{In}{incolor}{43}{\boxspacing}
\begin{Verbatim}[commandchars=\\\{\}]
\PY{n}{data\PYZus{}nul\PYZus{}PeopleVaccinatedPerHundred} \PY{o}{=} \PY{n}{data\PYZus{}nul}\PY{p}{[}\PY{p}{[}\PY{l+s+s1}{\PYZsq{}}\PY{l+s+s1}{people\PYZus{}vaccinated\PYZus{}per\PYZus{}hundred}\PY{l+s+s1}{\PYZsq{}}\PY{p}{]}\PY{p}{]}
\PY{n}{data\PYZus{}nul\PYZus{}PeopleVaccinatedPerHundred}\PY{o}{.}\PY{n}{head}\PY{p}{(}\PY{p}{)}
\end{Verbatim}
\end{tcolorbox}

            \begin{tcolorbox}[breakable, size=fbox, boxrule=.5pt, pad at break*=1mm, opacityfill=0]
\prompt{Out}{outcolor}{43}{\boxspacing}
\begin{Verbatim}[commandchars=\\\{\}]
   people\_vaccinated\_per\_hundred
0                            0.0
1                            NaN
2                            NaN
3                            NaN
4                            NaN
\end{Verbatim}
\end{tcolorbox}
        
    \begin{tcolorbox}[breakable, size=fbox, boxrule=1pt, pad at break*=1mm,colback=cellbackground, colframe=cellborder]
\prompt{In}{incolor}{45}{\boxspacing}
\begin{Verbatim}[commandchars=\\\{\}]
\PY{c+c1}{\PYZsh{} Фильтр для проверки заполнения пустых значений для одного из столбцов}
\PY{n}{indicator} \PY{o}{=} \PY{n}{MissingIndicator}\PY{p}{(}\PY{p}{)}
\PY{n}{mask\PYZus{}missing\PYZus{}values\PYZus{}only} \PY{o}{=} \PY{n}{indicator}\PY{o}{.}\PY{n}{fit\PYZus{}transform}\PY{p}{(}\PY{n}{data\PYZus{}nul\PYZus{}PeopleVaccinatedPerHundred}\PY{p}{)}
\PY{n}{mask\PYZus{}missing\PYZus{}values\PYZus{}only}
\end{Verbatim}
\end{tcolorbox}

            \begin{tcolorbox}[breakable, size=fbox, boxrule=.5pt, pad at break*=1mm, opacityfill=0]
\prompt{Out}{outcolor}{45}{\boxspacing}
\begin{Verbatim}[commandchars=\\\{\}]
array([[False],
       [ True],
       [ True],
       {\ldots},
       [False],
       [False],
       [False]])
\end{Verbatim}
\end{tcolorbox}
        
    Далее используем функцию SimpleImputer с использованием различных
показателей центра распределения (среднее значение, медиана, наиболее
часто встречающееся значение).

    \begin{tcolorbox}[breakable, size=fbox, boxrule=1pt, pad at break*=1mm,colback=cellbackground, colframe=cellborder]
\prompt{In}{incolor}{61}{\boxspacing}
\begin{Verbatim}[commandchars=\\\{\}]
\PY{n}{strategies}\PY{o}{=}\PY{p}{[}\PY{l+s+s1}{\PYZsq{}}\PY{l+s+s1}{mean}\PY{l+s+s1}{\PYZsq{}}\PY{p}{,} \PY{l+s+s1}{\PYZsq{}}\PY{l+s+s1}{median}\PY{l+s+s1}{\PYZsq{}}\PY{p}{,} \PY{l+s+s1}{\PYZsq{}}\PY{l+s+s1}{most\PYZus{}frequent}\PY{l+s+s1}{\PYZsq{}}\PY{p}{]}
\PY{k}{def} \PY{n+nf}{test\PYZus{}num\PYZus{}impute}\PY{p}{(}\PY{n}{strategy\PYZus{}param}\PY{p}{)}\PY{p}{:}
    \PY{n}{imp\PYZus{}num} \PY{o}{=} \PY{n}{SimpleImputer}\PY{p}{(}\PY{n}{strategy}\PY{o}{=}\PY{n}{strategy\PYZus{}param}\PY{p}{)}
    \PY{n}{data\PYZus{}nul\PYZus{}imp} \PY{o}{=} \PY{n}{imp\PYZus{}num}\PY{o}{.}\PY{n}{fit\PYZus{}transform}\PY{p}{(}\PY{n}{data\PYZus{}nul\PYZus{}PeopleVaccinatedPerHundred}\PY{p}{)}
    \PY{k}{return} \PY{n}{data\PYZus{}nul\PYZus{}imp}\PY{p}{[}\PY{n}{mask\PYZus{}missing\PYZus{}values\PYZus{}only}\PY{p}{]}
\end{Verbatim}
\end{tcolorbox}

    \begin{tcolorbox}[breakable, size=fbox, boxrule=1pt, pad at break*=1mm,colback=cellbackground, colframe=cellborder]
\prompt{In}{incolor}{62}{\boxspacing}
\begin{Verbatim}[commandchars=\\\{\}]
\PY{n}{strategies}\PY{p}{[}\PY{l+m+mi}{0}\PY{p}{]}\PY{p}{,} \PY{n}{test\PYZus{}num\PYZus{}impute}\PY{p}{(}\PY{n}{strategies}\PY{p}{[}\PY{l+m+mi}{0}\PY{p}{]}\PY{p}{)}
\end{Verbatim}
\end{tcolorbox}

            \begin{tcolorbox}[breakable, size=fbox, boxrule=.5pt, pad at break*=1mm, opacityfill=0]
\prompt{Out}{outcolor}{62}{\boxspacing}
\begin{Verbatim}[commandchars=\\\{\}]
('mean',
 array([9.35741607, 9.35741607, 9.35741607, {\ldots}, 9.35741607, 9.35741607,
        9.35741607]))
\end{Verbatim}
\end{tcolorbox}
        
    \begin{tcolorbox}[breakable, size=fbox, boxrule=1pt, pad at break*=1mm,colback=cellbackground, colframe=cellborder]
\prompt{In}{incolor}{63}{\boxspacing}
\begin{Verbatim}[commandchars=\\\{\}]
\PY{n}{strategies}\PY{p}{[}\PY{l+m+mi}{1}\PY{p}{]}\PY{p}{,} \PY{n}{test\PYZus{}num\PYZus{}impute}\PY{p}{(}\PY{n}{strategies}\PY{p}{[}\PY{l+m+mi}{1}\PY{p}{]}\PY{p}{)}
\end{Verbatim}
\end{tcolorbox}

            \begin{tcolorbox}[breakable, size=fbox, boxrule=.5pt, pad at break*=1mm, opacityfill=0]
\prompt{Out}{outcolor}{63}{\boxspacing}
\begin{Verbatim}[commandchars=\\\{\}]
('median', array([3.81, 3.81, 3.81, {\ldots}, 3.81, 3.81, 3.81]))
\end{Verbatim}
\end{tcolorbox}
        
    \begin{tcolorbox}[breakable, size=fbox, boxrule=1pt, pad at break*=1mm,colback=cellbackground, colframe=cellborder]
\prompt{In}{incolor}{65}{\boxspacing}
\begin{Verbatim}[commandchars=\\\{\}]
\PY{n}{strategies}\PY{p}{[}\PY{l+m+mi}{2}\PY{p}{]}\PY{p}{,} \PY{n}{test\PYZus{}num\PYZus{}impute}\PY{p}{(}\PY{n}{strategies}\PY{p}{[}\PY{l+m+mi}{2}\PY{p}{]}\PY{p}{)}
\end{Verbatim}
\end{tcolorbox}

            \begin{tcolorbox}[breakable, size=fbox, boxrule=.5pt, pad at break*=1mm, opacityfill=0]
\prompt{Out}{outcolor}{65}{\boxspacing}
\begin{Verbatim}[commandchars=\\\{\}]
('most\_frequent', array([0., 0., 0., {\ldots}, 0., 0., 0.]))
\end{Verbatim}
\end{tcolorbox}
        
    \hypertarget{ux43aux43eux434ux438ux440ux43eux432ux430ux43dux438ux435-ux43aux430ux442ux435ux433ux43eux440ux438ux430ux43bux44cux43dux44bux445-ux43fux440ux438ux437ux43dux430ux43aux43eux432.}{%
\subsubsection{Кодирование категориальных
признаков.}\label{ux43aux43eux434ux438ux440ux43eux432ux430ux43dux438ux435-ux43aux430ux442ux435ux433ux43eux440ux438ux430ux43bux44cux43dux44bux445-ux43fux440ux438ux437ux43dux430ux43aux43eux432.}}

Для исправления пропусков в категориальных признаках выберем другой
датасет, так как в предыдущем все пропуски в числовых столбцах.

    Тема данного датасета - тв шоу и сериалы сервиса Netflix. Подключим
данный датасет.

    \begin{tcolorbox}[breakable, size=fbox, boxrule=1pt, pad at break*=1mm,colback=cellbackground, colframe=cellborder]
\prompt{In}{incolor}{67}{\boxspacing}
\begin{Verbatim}[commandchars=\\\{\}]
\PY{n}{data2} \PY{o}{=} \PY{n}{pd}\PY{o}{.}\PY{n}{read\PYZus{}csv}\PY{p}{(}\PY{l+s+s1}{\PYZsq{}}\PY{l+s+s1}{netflix\PYZus{}titles.csv}\PY{l+s+s1}{\PYZsq{}}\PY{p}{,} \PY{n}{sep}\PY{o}{=}\PY{l+s+s2}{\PYZdq{}}\PY{l+s+s2}{,}\PY{l+s+s2}{\PYZdq{}}\PY{p}{)}
\PY{n}{data2}\PY{o}{.}\PY{n}{isnull}\PY{p}{(}\PY{p}{)}\PY{o}{.}\PY{n}{sum}\PY{p}{(}\PY{p}{)}
\end{Verbatim}
\end{tcolorbox}

            \begin{tcolorbox}[breakable, size=fbox, boxrule=.5pt, pad at break*=1mm, opacityfill=0]
\prompt{Out}{outcolor}{67}{\boxspacing}
\begin{Verbatim}[commandchars=\\\{\}]
show\_id            0
type               0
title              0
director        2389
cast             718
country          507
date\_added        10
release\_year       0
rating             7
duration           0
listed\_in          0
description        0
dtype: int64
\end{Verbatim}
\end{tcolorbox}
        
    Выделим один столбец, где удалим пропуски. Пусть это будет столбец
author.

    \begin{tcolorbox}[breakable, size=fbox, boxrule=1pt, pad at break*=1mm,colback=cellbackground, colframe=cellborder]
\prompt{In}{incolor}{68}{\boxspacing}
\begin{Verbatim}[commandchars=\\\{\}]
\PY{n}{cat\PYZus{}enc} \PY{o}{=} \PY{n}{pd}\PY{o}{.}\PY{n}{DataFrame}\PY{p}{(}\PY{p}{\PYZob{}}\PY{l+s+s1}{\PYZsq{}}\PY{l+s+s1}{c1}\PY{l+s+s1}{\PYZsq{}}\PY{p}{:}\PY{n}{data2}\PY{p}{[}\PY{l+s+s1}{\PYZsq{}}\PY{l+s+s1}{director}\PY{l+s+s1}{\PYZsq{}}\PY{p}{]}\PY{p}{\PYZcb{}}\PY{p}{)}
\PY{n}{cat\PYZus{}enc}
\end{Verbatim}
\end{tcolorbox}

            \begin{tcolorbox}[breakable, size=fbox, boxrule=.5pt, pad at break*=1mm, opacityfill=0]
\prompt{Out}{outcolor}{68}{\boxspacing}
\begin{Verbatim}[commandchars=\\\{\}]
                     c1
0                   NaN
1     Jorge Michel Grau
2          Gilbert Chan
3           Shane Acker
4        Robert Luketic
{\ldots}                 {\ldots}
7782        Josef Fares
7783        Mozez Singh
7784                NaN
7785                NaN
7786           Sam Dunn

[7787 rows x 1 columns]
\end{Verbatim}
\end{tcolorbox}
        
    \hypertarget{ux43aux43eux434ux438ux440ux43eux432ux430ux43dux438ux435-ux43aux430ux442ux435ux433ux43eux440ux438ux439-ux446ux435ux43bux43eux447ux438ux441ux43bux435ux43dux43dux44bux43cux438-ux437ux43dux430ux447ux435ux43dux438ux44fux43cux438.}{%
\paragraph{Кодирование категорий целочисленными
значениями.}\label{ux43aux43eux434ux438ux440ux43eux432ux430ux43dux438ux435-ux43aux430ux442ux435ux433ux43eux440ux438ux439-ux446ux435ux43bux43eux447ux438ux441ux43bux435ux43dux43dux44bux43cux438-ux437ux43dux430ux447ux435ux43dux438ux44fux43cux438.}}

Импортируем все необходимые библиотеки sklearn. Используем метод label
encoder.

    \begin{tcolorbox}[breakable, size=fbox, boxrule=1pt, pad at break*=1mm,colback=cellbackground, colframe=cellborder]
\prompt{In}{incolor}{69}{\boxspacing}
\begin{Verbatim}[commandchars=\\\{\}]
\PY{k+kn}{from} \PY{n+nn}{sklearn}\PY{n+nn}{.}\PY{n+nn}{preprocessing} \PY{k+kn}{import} \PY{n}{LabelEncoder}\PY{p}{,} \PY{n}{OneHotEncoder}
\end{Verbatim}
\end{tcolorbox}

    \begin{tcolorbox}[breakable, size=fbox, boxrule=1pt, pad at break*=1mm,colback=cellbackground, colframe=cellborder]
\prompt{In}{incolor}{70}{\boxspacing}
\begin{Verbatim}[commandchars=\\\{\}]
\PY{n}{le} \PY{o}{=} \PY{n}{LabelEncoder}\PY{p}{(}\PY{p}{)}
\PY{n}{cat\PYZus{}enc\PYZus{}le} \PY{o}{=} \PY{n}{le}\PY{o}{.}\PY{n}{fit\PYZus{}transform}\PY{p}{(}\PY{n}{cat\PYZus{}enc}\PY{p}{[}\PY{l+s+s1}{\PYZsq{}}\PY{l+s+s1}{c1}\PY{l+s+s1}{\PYZsq{}}\PY{p}{]}\PY{p}{)}
\end{Verbatim}
\end{tcolorbox}

    Просмотрим список имеющихся уникальных значений:

    \begin{tcolorbox}[breakable, size=fbox, boxrule=1pt, pad at break*=1mm,colback=cellbackground, colframe=cellborder]
\prompt{In}{incolor}{71}{\boxspacing}
\begin{Verbatim}[commandchars=\\\{\}]
\PY{n}{cat\PYZus{}enc}\PY{p}{[}\PY{l+s+s1}{\PYZsq{}}\PY{l+s+s1}{c1}\PY{l+s+s1}{\PYZsq{}}\PY{p}{]}\PY{o}{.}\PY{n}{unique}\PY{p}{(}\PY{p}{)}
\end{Verbatim}
\end{tcolorbox}

            \begin{tcolorbox}[breakable, size=fbox, boxrule=.5pt, pad at break*=1mm, opacityfill=0]
\prompt{Out}{outcolor}{71}{\boxspacing}
\begin{Verbatim}[commandchars=\\\{\}]
array([nan, 'Jorge Michel Grau', 'Gilbert Chan', {\ldots}, 'Josef Fares',
       'Mozez Singh', 'Sam Dunn'], dtype=object)
\end{Verbatim}
\end{tcolorbox}
        
    \begin{tcolorbox}[breakable, size=fbox, boxrule=1pt, pad at break*=1mm,colback=cellbackground, colframe=cellborder]
\prompt{In}{incolor}{72}{\boxspacing}
\begin{Verbatim}[commandchars=\\\{\}]
\PY{n}{np}\PY{o}{.}\PY{n}{unique}\PY{p}{(}\PY{n}{cat\PYZus{}enc\PYZus{}le}\PY{p}{)}
\end{Verbatim}
\end{tcolorbox}

            \begin{tcolorbox}[breakable, size=fbox, boxrule=.5pt, pad at break*=1mm, opacityfill=0]
\prompt{Out}{outcolor}{72}{\boxspacing}
\begin{Verbatim}[commandchars=\\\{\}]
array([   0,    1,    2, {\ldots}, 4047, 4048, 4049])
\end{Verbatim}
\end{tcolorbox}
        
    \begin{tcolorbox}[breakable, size=fbox, boxrule=1pt, pad at break*=1mm,colback=cellbackground, colframe=cellborder]
\prompt{In}{incolor}{73}{\boxspacing}
\begin{Verbatim}[commandchars=\\\{\}]
\PY{n}{le}\PY{o}{.}\PY{n}{inverse\PYZus{}transform}\PY{p}{(}\PY{p}{[}\PY{l+m+mi}{0}\PY{p}{]}\PY{p}{)}
\end{Verbatim}
\end{tcolorbox}

            \begin{tcolorbox}[breakable, size=fbox, boxrule=.5pt, pad at break*=1mm, opacityfill=0]
\prompt{Out}{outcolor}{73}{\boxspacing}
\begin{Verbatim}[commandchars=\\\{\}]
array(['A. L. Vijay'], dtype=object)
\end{Verbatim}
\end{tcolorbox}
        
    Попробуем метод one-hot encoding.

    \begin{tcolorbox}[breakable, size=fbox, boxrule=1pt, pad at break*=1mm,colback=cellbackground, colframe=cellborder]
\prompt{In}{incolor}{74}{\boxspacing}
\begin{Verbatim}[commandchars=\\\{\}]
\PY{n}{ohe} \PY{o}{=} \PY{n}{OneHotEncoder}\PY{p}{(}\PY{p}{)}
\PY{n}{cat\PYZus{}enc\PYZus{}ohe} \PY{o}{=} \PY{n}{ohe}\PY{o}{.}\PY{n}{fit\PYZus{}transform}\PY{p}{(}\PY{n}{cat\PYZus{}enc}\PY{p}{[}\PY{p}{[}\PY{l+s+s1}{\PYZsq{}}\PY{l+s+s1}{c1}\PY{l+s+s1}{\PYZsq{}}\PY{p}{]}\PY{p}{]}\PY{p}{)}
\end{Verbatim}
\end{tcolorbox}

    \begin{tcolorbox}[breakable, size=fbox, boxrule=1pt, pad at break*=1mm,colback=cellbackground, colframe=cellborder]
\prompt{In}{incolor}{75}{\boxspacing}
\begin{Verbatim}[commandchars=\\\{\}]
\PY{n}{cat\PYZus{}enc}\PY{o}{.}\PY{n}{shape}
\end{Verbatim}
\end{tcolorbox}

            \begin{tcolorbox}[breakable, size=fbox, boxrule=.5pt, pad at break*=1mm, opacityfill=0]
\prompt{Out}{outcolor}{75}{\boxspacing}
\begin{Verbatim}[commandchars=\\\{\}]
(7787, 1)
\end{Verbatim}
\end{tcolorbox}
        
    \begin{tcolorbox}[breakable, size=fbox, boxrule=1pt, pad at break*=1mm,colback=cellbackground, colframe=cellborder]
\prompt{In}{incolor}{76}{\boxspacing}
\begin{Verbatim}[commandchars=\\\{\}]
\PY{n}{cat\PYZus{}enc\PYZus{}ohe}\PY{o}{.}\PY{n}{shape}
\end{Verbatim}
\end{tcolorbox}

            \begin{tcolorbox}[breakable, size=fbox, boxrule=.5pt, pad at break*=1mm, opacityfill=0]
\prompt{Out}{outcolor}{76}{\boxspacing}
\begin{Verbatim}[commandchars=\\\{\}]
(7787, 4050)
\end{Verbatim}
\end{tcolorbox}
        
    \begin{tcolorbox}[breakable, size=fbox, boxrule=1pt, pad at break*=1mm,colback=cellbackground, colframe=cellborder]
\prompt{In}{incolor}{77}{\boxspacing}
\begin{Verbatim}[commandchars=\\\{\}]
\PY{n}{cat\PYZus{}enc\PYZus{}ohe}\PY{o}{.}\PY{n}{todense}\PY{p}{(}\PY{p}{)}\PY{p}{[}\PY{l+m+mi}{0}\PY{p}{:}\PY{l+m+mi}{10}\PY{p}{]}
\end{Verbatim}
\end{tcolorbox}

            \begin{tcolorbox}[breakable, size=fbox, boxrule=.5pt, pad at break*=1mm, opacityfill=0]
\prompt{Out}{outcolor}{77}{\boxspacing}
\begin{Verbatim}[commandchars=\\\{\}]
matrix([[0., 0., 0., {\ldots}, 0., 0., 1.],
        [0., 0., 0., {\ldots}, 0., 0., 0.],
        [0., 0., 0., {\ldots}, 0., 0., 0.],
        {\ldots},
        [0., 0., 0., {\ldots}, 0., 0., 0.],
        [0., 0., 0., {\ldots}, 0., 0., 0.],
        [0., 0., 0., {\ldots}, 0., 0., 0.]])
\end{Verbatim}
\end{tcolorbox}
        
    \begin{tcolorbox}[breakable, size=fbox, boxrule=1pt, pad at break*=1mm,colback=cellbackground, colframe=cellborder]
\prompt{In}{incolor}{78}{\boxspacing}
\begin{Verbatim}[commandchars=\\\{\}]
\PY{n}{pd}\PY{o}{.}\PY{n}{get\PYZus{}dummies}\PY{p}{(}\PY{n}{cat\PYZus{}enc}\PY{p}{)}\PY{o}{.}\PY{n}{head}\PY{p}{(}\PY{l+m+mi}{10}\PY{p}{)}
\end{Verbatim}
\end{tcolorbox}

            \begin{tcolorbox}[breakable, size=fbox, boxrule=.5pt, pad at break*=1mm, opacityfill=0]
\prompt{Out}{outcolor}{78}{\boxspacing}
\begin{Verbatim}[commandchars=\\\{\}]
   c1\_A. L. Vijay  c1\_A. Raajdheep  c1\_A. Salaam  c1\_A.R. Murugadoss  \textbackslash{}
0               0                0             0                   0
1               0                0             0                   0
2               0                0             0                   0
3               0                0             0                   0
4               0                0             0                   0
5               0                0             0                   0
6               0                0             0                   0
7               0                0             0                   0
8               0                0             0                   0
9               0                0             0                   0

   c1\_Aadish Keluskar  c1\_Aamir Bashir  c1\_Aamir Khan  c1\_Aanand Rai  \textbackslash{}
0                   0                0              0              0
1                   0                0              0              0
2                   0                0              0              0
3                   0                0              0              0
4                   0                0              0              0
5                   0                0              0              0
6                   0                0              0              0
7                   0                0              0              0
8                   0                0              0              0
9                   0                0              0              0

   c1\_Aaron Burns  c1\_Aaron Hancox, Michael McNamara  {\ldots}  \textbackslash{}
0               0                                  0  {\ldots}
1               0                                  0  {\ldots}
2               0                                  0  {\ldots}
3               0                                  0  {\ldots}
4               0                                  0  {\ldots}
5               0                                  0  {\ldots}
6               0                                  0  {\ldots}
7               0                                  0  {\ldots}
8               0                                  0  {\ldots}
9               0                                  0  {\ldots}

   c1\_Álex de la Iglesia  c1\_Álvaro Brechner  c1\_Álvaro Delgado-Aparicio L.  \textbackslash{}
0                      0                   0                              0
1                      0                   0                              0
2                      0                   0                              0
3                      0                   0                              0
4                      0                   0                              0
5                      0                   0                              0
6                      0                   0                              0
7                      0                   0                              0
8                      0                   0                              0
9                      0                   0                              0

   c1\_Álvaro Longoria, Gerardo Olivares  c1\_Ángel Gómez Hernández  \textbackslash{}
0                                     0                         0
1                                     0                         0
2                                     0                         0
3                                     0                         0
4                                     0                         0
5                                     0                         0
6                                     0                         0
7                                     0                         0
8                                     0                         0
9                                     0                         0

   c1\_Çagan Irmak  c1\_Ísold Uggadóttir  c1\_Óskar Thór Axelsson  \textbackslash{}
0               0                    0                       0
1               0                    0                       0
2               0                    0                       0
3               0                    0                       0
4               0                    0                       0
5               0                    0                       0
6               0                    0                       0
7               0                    0                       0
8               0                    0                       0
9               0                    0                       0

   c1\_Ömer Faruk Sorak  c1\_Şenol Sönmez
0                    0                0
1                    0                0
2                    0                0
3                    0                0
4                    0                0
5                    0                0
6                    0                0
7                    0                0
8                    0                0
9                    0                0

[10 rows x 4049 columns]
\end{Verbatim}
\end{tcolorbox}
        
    \hypertarget{ux43cux430ux441ux448ux442ux430ux431ux438ux440ux43eux432ux430ux43dux438ux435-ux434ux430ux43dux43dux44bux445.}{%
\subsubsection{Масштабирование
данных.}\label{ux43cux430ux441ux448ux442ux430ux431ux438ux440ux43eux432ux430ux43dux438ux435-ux434ux430ux43dux43dux44bux445.}}

Данная операция неоюходима для того, чтобы привести диапазоны величин,
имеющих пропуски, к примерно одной области. Подберем датасет, подходящий
для этой операции.

    \begin{tcolorbox}[breakable, size=fbox, boxrule=1pt, pad at break*=1mm,colback=cellbackground, colframe=cellborder]
\prompt{In}{incolor}{83}{\boxspacing}
\begin{Verbatim}[commandchars=\\\{\}]
\PY{n}{data3} \PY{o}{=} \PY{n}{pd}\PY{o}{.}\PY{n}{read\PYZus{}csv}\PY{p}{(}\PY{l+s+s1}{\PYZsq{}}\PY{l+s+s1}{oasis\PYZus{}cross\PYZhy{}sectional.csv}\PY{l+s+s1}{\PYZsq{}}\PY{p}{,} \PY{n}{sep}\PY{o}{=}\PY{l+s+s2}{\PYZdq{}}\PY{l+s+s2}{,}\PY{l+s+s2}{\PYZdq{}}\PY{p}{)}
\PY{n}{data3}\PY{o}{.}\PY{n}{describe}\PY{p}{(}\PY{p}{)}
\end{Verbatim}
\end{tcolorbox}

            \begin{tcolorbox}[breakable, size=fbox, boxrule=.5pt, pad at break*=1mm, opacityfill=0]
\prompt{Out}{outcolor}{83}{\boxspacing}
\begin{Verbatim}[commandchars=\\\{\}]
              Age        Educ         SES       MMSE         CDR         eTIV  \textbackslash{}
count  436.000000  235.000000  216.000000  235.00000  235.000000   436.000000
mean    51.357798    3.178723    2.490741   27.06383    0.285106  1481.919725
std     25.269862    1.311510    1.120593    3.69687    0.383405   158.740866
min     18.000000    1.000000    1.000000   14.00000    0.000000  1123.000000
25\%     23.000000    2.000000    2.000000   26.00000    0.000000  1367.750000
50\%     54.000000    3.000000    2.000000   29.00000    0.000000  1475.500000
75\%     74.000000    4.000000    3.000000   30.00000    0.500000  1579.250000
max     96.000000    5.000000    5.000000   30.00000    2.000000  1992.000000

             nWBV         ASF     Delay
count  436.000000  436.000000  20.00000
mean     0.791670    1.198894  20.55000
std      0.059937    0.128682  23.86249
min      0.644000    0.881000   1.00000
25\%      0.742750    1.111750   2.75000
50\%      0.809000    1.190000  11.00000
75\%      0.842000    1.284250  30.75000
max      0.893000    1.563000  89.00000
\end{Verbatim}
\end{tcolorbox}
        
    Пропуски содержатся в поле eTIV и там же можно нормализовать данные
сместив их к диапазону от нуля до единицы.

    Импортируем необходимые библиотеки. Импортируем библиотеки.

    \begin{tcolorbox}[breakable, size=fbox, boxrule=1pt, pad at break*=1mm,colback=cellbackground, colframe=cellborder]
\prompt{In}{incolor}{88}{\boxspacing}
\begin{Verbatim}[commandchars=\\\{\}]
\PY{k+kn}{from} \PY{n+nn}{sklearn}\PY{n+nn}{.}\PY{n+nn}{preprocessing} \PY{k+kn}{import} \PY{n}{MinMaxScaler}\PY{p}{,} \PY{n}{StandardScaler}\PY{p}{,} \PY{n}{Normalizer}
\end{Verbatim}
\end{tcolorbox}

    \begin{tcolorbox}[breakable, size=fbox, boxrule=1pt, pad at break*=1mm,colback=cellbackground, colframe=cellborder]
\prompt{In}{incolor}{90}{\boxspacing}
\begin{Verbatim}[commandchars=\\\{\}]
\PY{n}{sc1} \PY{o}{=} \PY{n}{MinMaxScaler}\PY{p}{(}\PY{p}{)}
\PY{n}{sc1\PYZus{}data} \PY{o}{=} \PY{n}{sc1}\PY{o}{.}\PY{n}{fit\PYZus{}transform}\PY{p}{(}\PY{n}{data3}\PY{p}{[}\PY{p}{[}\PY{l+s+s1}{\PYZsq{}}\PY{l+s+s1}{eTIV}\PY{l+s+s1}{\PYZsq{}}\PY{p}{]}\PY{p}{]}\PY{p}{)}
\end{Verbatim}
\end{tcolorbox}

    Рассмотрим сходный график.

    \begin{tcolorbox}[breakable, size=fbox, boxrule=1pt, pad at break*=1mm,colback=cellbackground, colframe=cellborder]
\prompt{In}{incolor}{93}{\boxspacing}
\begin{Verbatim}[commandchars=\\\{\}]
\PY{n}{plt}\PY{o}{.}\PY{n}{hist}\PY{p}{(}\PY{n}{data3}\PY{p}{[}\PY{l+s+s1}{\PYZsq{}}\PY{l+s+s1}{eTIV}\PY{l+s+s1}{\PYZsq{}}\PY{p}{]}\PY{p}{,} \PY{l+m+mi}{50}\PY{p}{)}
\PY{n}{plt}\PY{o}{.}\PY{n}{show}\PY{p}{(}\PY{p}{)}
\end{Verbatim}
\end{tcolorbox}

    \begin{center}
    \adjustimage{max size={0.9\linewidth}{0.9\paperheight}}{Лабораторная работа №2_files/Лабораторная работа №2_55_0.png}
    \end{center}
    { \hspace*{\fill} \\}
    
    И рассмотрим полученный график.

    \begin{tcolorbox}[breakable, size=fbox, boxrule=1pt, pad at break*=1mm,colback=cellbackground, colframe=cellborder]
\prompt{In}{incolor}{92}{\boxspacing}
\begin{Verbatim}[commandchars=\\\{\}]
\PY{n}{plt}\PY{o}{.}\PY{n}{hist}\PY{p}{(}\PY{n}{sc1\PYZus{}data}\PY{p}{,} \PY{l+m+mi}{50}\PY{p}{)}
\PY{n}{plt}\PY{o}{.}\PY{n}{show}\PY{p}{(}\PY{p}{)}
\end{Verbatim}
\end{tcolorbox}

    \begin{center}
    \adjustimage{max size={0.9\linewidth}{0.9\paperheight}}{Лабораторная работа №2_files/Лабораторная работа №2_57_0.png}
    \end{center}
    { \hspace*{\fill} \\}
    
    Также можно использовать масштабирование данных на основе Z-оценки,
позволяющее сместить данные в диапазон от 0 до 3. Проверим этот метод на
том же признаке.

    \begin{tcolorbox}[breakable, size=fbox, boxrule=1pt, pad at break*=1mm,colback=cellbackground, colframe=cellborder]
\prompt{In}{incolor}{95}{\boxspacing}
\begin{Verbatim}[commandchars=\\\{\}]
\PY{n}{sc2} \PY{o}{=} \PY{n}{StandardScaler}\PY{p}{(}\PY{p}{)}
\PY{n}{sc2\PYZus{}data} \PY{o}{=} \PY{n}{sc2}\PY{o}{.}\PY{n}{fit\PYZus{}transform}\PY{p}{(}\PY{n}{data3}\PY{p}{[}\PY{p}{[}\PY{l+s+s1}{\PYZsq{}}\PY{l+s+s1}{eTIV}\PY{l+s+s1}{\PYZsq{}}\PY{p}{]}\PY{p}{]}\PY{p}{)}
\end{Verbatim}
\end{tcolorbox}

    \begin{tcolorbox}[breakable, size=fbox, boxrule=1pt, pad at break*=1mm,colback=cellbackground, colframe=cellborder]
\prompt{In}{incolor}{96}{\boxspacing}
\begin{Verbatim}[commandchars=\\\{\}]
\PY{n}{plt}\PY{o}{.}\PY{n}{hist}\PY{p}{(}\PY{n}{sc2\PYZus{}data}\PY{p}{,} \PY{l+m+mi}{50}\PY{p}{)}
\PY{n}{plt}\PY{o}{.}\PY{n}{show}\PY{p}{(}\PY{p}{)}
\end{Verbatim}
\end{tcolorbox}

    \begin{center}
    \adjustimage{max size={0.9\linewidth}{0.9\paperheight}}{Лабораторная работа №2_files/Лабораторная работа №2_60_0.png}
    \end{center}
    { \hspace*{\fill} \\}
    

    % Add a bibliography block to the postdoc
    
    
    
\end{document}
